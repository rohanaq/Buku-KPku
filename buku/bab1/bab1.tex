\chapter{PENDAHULUAN}

\section{Latar Belakang}
\tab Saat ini Telkomsel adalah operator seluler terbesar di Indonesia dengan 178 juta pelanggan dan untuk melayani pelanggannya yang tersebar di seluruh Indonesia, termasuk juga di daerah terpencil dan pulau terluar serta daerah perbatasan negara, Telkomsel menggelar lebih dari 146 ribu BTS. \\
\tab Untuk memberikan layanan yang prima kepada masyarakat di dalam menikmati gaya hidup digital (\textit{digital lifestyle}), Telkomsel turut membangun ekosistem digital di tanah air melalui berbagai upaya pengembangan DNA (\textit{Device}, \textit{Network} dan \textit{Applications}), yang diharapkan akan mempercepat terbentuknya masyarakat digital Indonesia.\\
\tab Sebelumnya, untuk pencatatan hasil pencapaian \textit{revenue} hanya dilakukan dengan menggunakan \textit{tools} berupa Microsoft Excel dan media \textit{WhatsApp} maupun \textit{E-mail} untuk menyebarkan hasilnya. Oleh karena itu, dibutuhkan suatu sistem untuk memantau pencapai \textit{revenue digital services} yang disediakan oleh Telkomsel\cite{telkomsel}.

\section{Tujuan}
Pembuatan Sistem TPORT ini bertujuan untuk:
\begin{enumerate}
\item Pengolahan data langsung melalui aplikasi web yang dibangun
\item Tersedianya informasi laporan hasil pencapaian tanpa harus disebarkan melalui \textit{WhatsApp} atau \textit{E-mail}
\item Melakukan efisiensi kerja 
\end{enumerate}

\section{Manfaat}
Manfaat dari pembuatan Sistem TPORT adalah sebagai berikut:

\begin{enumerate}
	\item Memudahkan proses pemantauan pencapaian \textit{revenue}
	\item Memudahkan penyebaran hasil karena sudah bisa dibuka pada aplikasi web yang dibangun
\end{enumerate}

\section{Rumusan Masalah}
Rumusan Masalah dari Kerja Praktik ini adalah sebagai berikut:

\begin{enumerate}
	\item Bagaimana menciptakan aplikasi web yang mudah dimengerti oleh pengguna?
	\item Bagaimana menciptakan aplikasi web TPORT sehingga mempermudah proses pemantauan pencapaian \textit{revenue}?
\end{enumerate}

\section{Lokasi dan Waktu Kerja Praktik}
\tab Lokasi kerja praktik berada di Grapari Pemuda dengan alamat Jalan Pemuda No 27-31, Embong Kaliasin, Genteng, Surabaya.\\
\tab Adapun kerja praktik dimulai pada tanggal 8 Januari 2018 hingga 9 Februari 2018 dengan hari kerja Senin sampai Jumat pukul 08.00 sampai dengan pukul 17.00 WIB (8 jam kerja dan 1 jam istirahat).

\section{Metodologi}
Metodologi dalam pembuatan buku Kerja Praktik ini meliputi:
\begin{enumerate}
	\item \textbf{Perumusan Masalah}\\
	Untuk mengetahui domain dan fungsionalitas, dijelaskan secara rinci bagaimana sistem yang harus dibuat. Penjelasan oleh pembimbing kerja praktik kali ini menghasilkan beberapa catatan mengenai gambaran cara kerja sistem dan rincian kebutuhan sistem. Setelah mendapatkan gambaran sistem, diskusi lebih lanjut dilakukan guna menentukan DBMS, bahasa pemrograman, dan framework yang dipakai dalam pembuatan sistem.
	
	\item \textbf{Studi Literatur}\\
	Pada tahap ini, setelah ditentukannya DBMS, bahasa pemrograman sampai dengan framework yang digunakan, dilakukan studi literatur lanjut mengenai bagaimana penggunaannya dalam membangun sistem sesuai yang diharapkan.
	
	Secara garis besar, untuk membuat TPORT digunakan bahasa pemrograman HTML, CSS, Javascript dan PHP untuk \textit{back end} sistem, serta DBMS MySQL sebagai penyimpanan data pencapaian \textit{revenue}, yang dikemas	melalui framework Laravel.
	
	\item \textbf{Analisis dan Perancangan Sistem}\\
	Pada tahap ini akan dijelaskan tentang analisis serta perancangan sistem yang akan dibangun oleh penulis.
	
	\item \textbf{Implementasi Sistem}\\
	Implementasi merupakan tahap pembangunan rancangan. Pada tahap ini merealisasikan apa yang terjadi pada tahap sebelumnya, sehingga sesuai dengan apa yang telah direncanakan.
	
	\item \textbf{Pengujian dan Evaluasi}\\
	Pada tahap ini dilakukan uji coba pada aplikasi yang telah diimplementasikan. Tahap ini bermaksud untuk mengevaluasi kesesuaian sistem dan aplikasi yang dibuat apakah dapat dilakukan dengan lancar atau tidak. Selain itu juga untuk mencari masalah yang mungkin timbul dan tidak lupa mengadakan perbaikan jika terdapat kesalahan.
	
	\item \textbf{Kesimpulan dan Saran}\\
	Pengujian yang dilakukan ini telah memenuhi syarat yang diinginkan, dan berjalan dengan baik dan lancar.
\end{enumerate}

\section{Sistematika Laporan}
Laporan Kerja Praktik ini terbagi menjadi 7 bab dengan rincian sebagai berikut:
	\begin{enumerate}
	\item BAB I: PENDAHULUAN
	
	Bab ini berisi latar belakang, tujuan, manfaat, rumusan masalah, lokasi dan waktu kerja praktik, metodologi dan sistematika laporan.
		
	\item BAB II: PROFIL PERUSAHAAN
		
	Bab ini berisi gambaran umum PT Telekomunikasi Seluler Indonesia, mulai dari sejarah, tujuan, visi dan misi perusahaan, dan divisi tempat kerja praktik dilakukan.
		
	\item BAB III: TINJAUAN PUSTAKA
		
	Bab ini berisi dasar teori dari metode/teknologi yang digunakan dalam menyelesaikan proyek kerja praktik.
		
	\item BAB IV: ANALISIS DAN PERANCANGAN SISTEM
		
	Bab ini dijelaskan mengenai desain antarmuka aplikasi.
		
	\item BAB V: IMPLEMENTASI SISTEM
		
	Bab ini berisi uraian tahap-tahap yang dilakukan untuk proses implementasi aplikasi.
		
	\item BAB VI: HASIL DAN UJI COBA
	
	Bab ini berisi hasil uji coba dan evaluasi dari perangkat lunak yang telah dikembangkan selama pelaksanaan kerja praktik.
	
	\item KESIMPULAN DAN SARAN
	
	Bab ini berisi kesimpulan dan saran yang didapat dari proses pelaksanaan kerja praktik.
	\end{enumerate}

\cleardoublepage
