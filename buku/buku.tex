\documentclass[11pt,a5paper,twoside,bahasa,listof=nochaptergap]{book}
\usepackage[tmargin=2.5cm,bmargin=2.5cm,lmargin=2.5cm,rmargin=2.0cm]{geometry}
\setcounter{secnumdepth}{5}
\usepackage[titletoc]{appendix}
\usepackage{csvsimple,longtable,booktabs}
\usepackage{listings}
\usepackage{diagbox}
\usepackage{csquotes}
\usepackage{tikz}
\usepackage{amsmath}
\usepackage[titles]{tocloft}

\begingroup
\makeatletter
\let\newcounter\@gobble\let\setcounter\@gobbletwo
\globaldefs\@ne \let\c@loldepth\@ne
\newlistof{listings}{lol}{\lstlistlistingname}
\newlistentry{lstlisting}{lol}{0}
\makeatother
\endgroup

\cftsetindents{lstlisting}{0in}{1.3in}

\lstset{ %
	numbers=left,
	xleftmargin=2mm,
	xrightmargin=2mm,
	frame=single,
	framesep=2mm,
	basicstyle=\footnotesize\ttfamily,
	framexleftmargin=2em,
	captionpos=b,
	belowcaptionskip=4pt,
	frame=single,
	breaklines=true,
	postbreak=\raisebox{0ex}[0ex][0ex]{\ensuremath{\color{red}\hookrightarrow\space}}
}
\usepackage{color, colortbl}
\definecolor{LightCyan}{rgb}{0.88,1,1}
\usepackage{rotating}
\usepackage{mdframed}
\usepackage{clrscode3e}
\usepackage[parfill]{parskip}
\usepackage{graphics}
\usepackage{fontspec}
\setsansfont{Trebuchet MS}
\setmainfont{Times New Roman}
\setmonofont{Courier New}
\usepackage{tocbibind}
\usepackage{enumitem}
\usepackage{pdflscape}
\usepackage{minted}
\usepackage{multirow}
\usepackage{graphicx}
\usepackage[verbose]{wrapfig}
\usepackage{changepage}
\usepackage{eso-pic}
\usepackage{ragged2e}
\usepackage{xesearch}
\usepackage[bahasa]{babel}
\usepackage[breaklinks=true]{hyperref}
\usepackage[font=small]{caption}
\usepackage{enumitem}
\setlist{nolistsep}
\usepackage{float}
\usepackage{longtable}
\usepackage{array,etoolbox}
\usepackage{amssymb}
\preto\tabular{\setcounter{magicrownumbers}{0}}
\preto\longtable{\setcounter{magicrownumbers}{0}}
\newcounter{magicrownumbers}
\newcommand\rownumber{\stepcounter{magicrownumbers}\arabic{magicrownumbers}}

\newcommand\nc[1]{%
	\multicolumn{1}{c}{#1}%
}

\newcommand\tab[1][1cm]{\hspace*{#1}}

\usepackage{setspace}
\singlespacing
\usepackage{fancyhdr}
\fancyhf{}
\renewcommand{\headrulewidth}{0pt}
\lhead[\thepage]{}
\rhead[]{\thepage}
\pagestyle{fancy}

\usepackage{titlesec}
\titleformat{\chapter}[display]{\filcenter\fontsize{11}{11}\bfseries}{\chaptername \ \thechapter}{0pt}{\filcenter\fontsize{11}{11}\bfseries\uppercase}
\titlespacing*{\chapter}{0pt}{-0.5cm}{40pt}
\titlespacing*{\section}{0pt}{11pt}{0pt}
\titlespacing*{\subsection}{0pt}{11pt}{0pt}
\titlespacing*{\subsubsection}{0pt}{11pt}{0pt}
\titleformat*{\section}{\fontsize{11}{11}\bfseries}
\titleformat*{\subsection}{\fontsize{11}{11}\bfseries}
\titleformat*{\subsubsection}{\fontsize{11}{11}\bfseries}


\addto\captionsbahasa{%
	\renewcommand\bibname{DAFTAR PUSTAKA}%
	\renewcommand\contentsname{DAFTAR ISI}%
	\renewcommand\listtablename{DAFTAR TABEL}%
	\renewcommand\listfigurename{DAFTAR GAMBAR}%
	\renewcommand\chaptername{BAB}%
}

\setlength{\parindent}{1cm}

\usepackage{chngcntr}
\renewcommand{\lstlistingname}{Kode Sumber}
\renewcommand{\lstlistlistingname}{DAFTAR KODE SUMBER}
\renewcommand*\thechapter{\arabic{chapter}}
\renewcommand\cftlstlistingpresnum{Kode Sumber }
\newcommand{\var}[2]{\newcommand{#1}{#2}}
\var{\judul}{Rancang Bangun Aplikasi "TPort" Sebagai Media Informasi Pencapaian Revenue Menggunakan Kerangka Kerja Laravel}
\var{\judulEnglish}{}

\var{\penulis}{Rohana Qudus }
\var{\nrp}{5115 100 045 }
\var{\penulisDua}{Rafi R. Ramadhan }
\var{\nrpDua}{5115 100 158 }
\var{\jurusan}{Informatika }
\var{\jurusanEnglish}{Informatics Department }
\var{\fakultas}{Teknologi Informasi dan Komunikasi }
\var{\fakultasEnglish}{Information Technology and Communication }
\var{\prodi}{S-1 }
\var{\pembimbingJurusan}{Ir. Muchammad Husni, M.Kom.}
\var{\nipPembimbingSatu}{19600221 198403 1 001}
\var{\pembimbingLapangan}{Herbintarto}
\var{\nipPembimbingDua}{195701011983031004}


\usepackage{caption}
\captionsetup[table]{labelsep=space}
\captionsetup[figure]{labelsep=space}
\captionsetup[lstlisting]{labelsep=space}

\setlength\cftparskip{-2pt}
\setlength\cftbeforechapskip{0pt}
\setlength{\lineskip}{0pt}

% Pemenggalan Tambahan
%english
%\hyphenation{ver-tex}
%\hyphenation{bridge}
%\hyphenation{block}
%\hyphenation{weight}
%\hyphenation{dis-joint}
%\hyphenation{par-ent}
%\hyphenation{root}
%\hyphenation{node}
%\hyphenation{set}
%\hyphenation{bridge-block}
%\hyphenation{con-nected}
%\hyphenation{com-po-nent}
%\hyphenation{undi-rected}
%\hyphenation{di-rected}
%\hyphenation{par-tially}
%\hyphenation{fully}
%\hyphenation{query}
%\hyphenation{dataset}

%bahasa
\hyphenation{me-nge-ta-hui}
\hyphenation{me-nya-ta-kan}
\hyphenation{meng-i-ni-si-al-i-sa-si}
\hyphenation{meng-gu-na-kan}
\hyphenation{me-la-ku-kan}
\hyphenation{me-li-bat-kan}
\hyphenation{permasalahan}

\makeatletter
\def\emptypage@emptypage{%
	\begin{center}
		\emph{Halaman ini sengaja dikosongkan}
	\end{center}
	\newpage%
}%
\def\cleardoublepage{%
	\clearpage%
	\if@twoside%
	\ifodd\c@page%
	% do nothing
	\else%
	\emptypage@emptypage%
	\fi%
	\fi%
}%
\makeatother
\raggedbottom


\renewcommand{\cftchapleader}{\cftdotfill{\cftdotsep}}
\renewcommand{\cftchappresnum}{BAB }
\renewcommand{\cfttabpresnum}{Tabel }
\cftsetindents{tab}{1em}{4.7em}
\renewcommand{\cftfigpresnum}{Gambar }
\cftsetindents{fig}{1em}{5.5em}

\cftsetindents{chapter}{0em}{3.6em}      % set amount of 
\cftsetindents{section}{2em}{2em}

\renewcommand{\thechapter}{\Roman{chapter}}
\renewcommand{\thesection}{\arabic{chapter}.\arabic{section}}
\renewcommand{\thesubsection}{\arabic{chapter}.\arabic{section}.\arabic{subsection}}
\renewcommand{\thefigure}{\arabic{chapter}.\arabic{figure}}
\renewcommand{\thetable}{\arabic{chapter}.\arabic{table}}

\newcommand{\insertfigure}{\begin{figure}\caption{A figure}\end{figure}}
\usepackage{etoolbox}% http://ctan.org/pkg/etoolbox
\makeatletter
% \patchcmd{<cmd>}{<search>}{<replace>}{<succes>}{<failure>}
\patchcmd{\@chapter}{\addtocontents{lof}{\protect\addvspace{10\p@}}}{}{}{}
\patchcmd{\@chapter}{\addtocontents{lot}{\protect\addvspace{10\p@}}}{}{}{}
\makeatother

\makeatletter
\g@addto@macro\appendix{%
	\addtocontents{toc}{%
		\protect\renewcommand{\protect\cftchappresnum}{}%
	}%
}
\makeatother


\renewcommand{\cftchapaftersnumb}{\hspace{0em}}

% Table caption above table
\floatstyle{plaintop}
\restylefloat{table}

% Centering table caption
\usepackage[justification=centering]{caption}
% Prevent hyphenation
\usepackage[none]{hyphenat}

\usepackage{mfirstuc}

\usepackage{tikz}

\begin{document} \sloppy
	% To prevent lstlisting from roman numbering
	\renewcommand{\thelstlisting}{\arabic{chapter}.\arabic{lstlisting}}
	\renewcommand{\theequation}{\arabic{chapter}.\arabic{equation}}
	
	% To remove spacing between chapter in list of figure and list of table
	\addtocontents{lof}{\protect\renewcommand*\protect\addvspace[1]{}}
	\addtocontents{lot}{\protect\renewcommand*\protect\addvspace[1]{}}
	
	% \setlength{\abovecaptionskip}{-12.75pt}
	
	\title {\judul}
	\author {\penulis}
	
	\frontmatter
	\addcontentsline{toc}{chapter}{SAMPUL}
	\newpage
	\newgeometry{top=7cm,left=2cm,bottom=2cm}

	\sffamily
	\thispagestyle{empty}
	\color{white}
	{ \noindent KERJA PRAKTIK - KI141330 }\\*[10pt] 
	{\large\textbf{\MakeUppercase{\judul}}} \\*[32pt]
	\\
	\\
	\MakeUppercase{\penulis} \\*
	NRP \nrp \\*[10pt]
	\MakeUppercase{\penulisDua} \\*
	NRP \nrpDua \\*[10pt]
	Dosen Pembimbing Jurusan \\*
	\pembimbingJurusan \\*[10pt]
	Pembimbing Lapangan \\*
	\pembimbingLapangan \\*[10pt]
	DEPARTEMEN \MakeUppercase{\jurusan} \\*
	Fakultas \fakultas \\*
	Institut Teknologi Sepuluh Nopember \\*
	Surabaya, 2018 \\*
	\AddToShipoutPictureBG*{\includegraphics[width=\paperwidth,height=\paperheight]{sampul/sampul.png}}
	\rmfamily
	\normalsize
	\restoregeometry
	\color{black}
	\cleardoublepage
	
\newpage
	\newgeometry{top=7cm,left=2cm,bottom=2cm}

	\sffamily
	\thispagestyle{empty}
	{ \noindent KERJA PRAKTIK - KI141330 }\\*[10pt] 
	{\large\textbf{\MakeUppercase{\judul}}} \\*[32pt]
	\\
	\\
	\MakeUppercase{\penulis} \\*
	NRP \nrp \\*[10pt]
	\MakeUppercase{\penulisDua} \\*
	NRP \nrpDua \\*[10pt]
	Dosen Pembimbing Jurusan \\*
	\pembimbingJurusan \\*[10pt]
	Pembimbing Lapangan \\*
	\pembimbingLapangan \\*[10pt]
	DEPARTEMEN \MakeUppercase{\jurusan} \\*
	Fakultas \fakultas \\*
	Institut Teknologi Sepuluh Nopember \\*
	Surabaya, 2018 \\*
	\AddToShipoutPictureBG*{\includegraphics[width=\paperwidth,height=\paperheight]{sampul/sampulWhite.png}}
	\rmfamily
	\normalsize
	\restoregeometry
	\color{black}
	\cleardoublepage

	\addcontentsline{toc}{chapter}{LEMBAR PENGESAHAN}
	\newpage
	\thispagestyle{plain}
	\begin{centering}
		\textbf{\MakeUppercase{\judul}} \\*[10pt]
		\textbf{\large{KERJA PRAKTIK}} \\*
		
		Oleh: \\*
		\textbf{\penulis} \\*
		NRP: \nrp \\*[20pt]
		\textbf{\penulisDua} \\*
		NRP: \nrpDua \\*[20pt]
	\end{centering}

	{\noindent Disetujui oleh Pembimbing Kerja Praktik:}\\*[20pt]         
	\pembimbingJurusan \hfill \hfill .......................... \\
		NIP: \nipPembimbingSatu \hfill \hfill (Pembimbing Jurusan) \\*[20pt]
		
	{\noindent \pembimbingLapangan  \hfill \hfill ...........................}  \\
	.\hfill \hfill (Pembimbing Lapangan) \\*[20pt] 

	\begin{centering}
		\textbf{SURABAYA} \\*
		\textbf{FEBRUARI 2018} \\*
	\end{centering}
	\cleardoublepage
	% INDONESIAN ABSTRAK
\addcontentsline{toc}{chapter}{ABSTRAK}
\thispagestyle{plain}
\begin{centering}
\textbf{\MakeUppercase{\judul}}
\end{centering}

\begin{tabular}{ll}
Nama  & : \MakeUppercase{\penulis} \\
NRP & : \nrp \\
Nama  & : \MakeUppercase{\penulisDua} \\
NRP & : \nrpDua \\
Departemen  & : Departemen \jurusan FTIK-ITS \\
Pembimbing Jurusan  & : Ir. Muchammad Husni, M.Kom. \\
Pembimbing Lapangan  & : \pembimbingLapangan
\end{tabular}
\\*[20pt]
\begin{centering}
\textbf{Abstrak}
\end{centering}
\itshape
% BEGIN
\\
\indent Dengan kemajuan teknologi di zaman sekarang, BRI menjadi satu-satunya Bank di dunia yang memiliki satelit sendiri. Oleh karenanya, dilakukan relokasi ATM untuk diarahkan langsung menuju satelit milik BRI. Untuk keberlangsungan kegiatan relokasi ATM, diperlukan sebuah sistem informasi sebagai wadah kontrol pengerjaan proyek relokasi ATM BRI.\\
Pada laporan Kerja Praktik kali ini, penulis akan menguraikan secara garis besar pengerjaan aplikasi \textit{Monitoring} SIK yang menggunakan bahasa pemrograman HTML, CSS, JavaScript serta PHP.\\
Berdasarkan hasil uji coba dan evaluasi menunjukkan bahwa aplikasi \textit{Monitoring} SIK yang dibuat telah berhasil memenuhi kebutuhan informasi yang dibutuhkan dalam memantau pengerjaan proyek relokasi.
% END
\rm \\
\textbf{Kata Kunci: \textit{monitoring}, SIK, relokasi}


\cleardoublepage

	\chapter{KATA PENGANTAR}

\indent\indent	Puji syukur penulis panjatkan kepada Allah SWT atas pimpinan, penyertaan, dan karunia-Nya sehingga penulis dapat menyelesaikan salah satu kewajiban penulis sebagai mahasiswa Departemen Informatika ITS yaitu Kerja Praktik yang berjudul:
\begin{center}
	\textbf{\MakeUppercase{\judul}}.
\end{center}

\indent Penulis menyadari bahwa masih banyak kekurangan baik dalam pelaksanaan Kerja Praktik maupun penyusunan buku laporan ini. Namun penulis berharap buku laporan ini dapat menambah wawasan pembaca dan dapat menjadi sumber referensi.\\
\indent Melalui buku laporan ini penulis juga ingin menyampaikan rasa terima kasih kepada orang-orang yang telah membantu dalam penyusunan laporan Kerja Praktik baik secara langsung maupun tidak langsung antara lain:
\begin{enumerate}
\item Kedua orang tua penulis.
\item Bapak Ir. Muchammad Husni, M.Kom., selaku dosen pembimbing kerja praktik selama kerja praktik berlangsung.
\item Bapak Radityo Anggoro, S.Kom., M.Sc., Dr.Eng., selaku koordinator kerja praktik.
\item Bapak Herbintarto selaku pembimbing lapangan selama kerja praktik berlangsung.
\item Mas Zikril selaku pembimbing lapangan selama kerja praktik berlangsung.\\
\end{enumerate}

\hfill Surabaya, Februari 2018 \\ \\
\begin{flushright}
\hfill{\penulis} \\
\hfill{\penulisDua}
\end{flushright}
\cleardoublepage

	\tableofcontents
	\cleardoublepage
	\listoftables
	\cleardoublepage
	\listoffigures
	\cleardoublepage
	\lstlistoflistings
	\cleardoublepage
	
	\mainmatter
	\chapter{PENDAHULUAN}

\section{Latar Belakang}
\tab Saat ini Telkomsel adalah operator seluler terbesar di Indonesia dengan 178 juta pelanggan dan untuk melayani pelanggannya yang tersebar di seluruh Indonesia, termasuk juga di daerah terpencil dan pulau terluar serta daerah perbatasan negara, Telkomsel menggelar lebih dari 146 ribu BTS. \\
\tab Untuk memberikan layanan yang prima kepada masyarakat di dalam menikmati gaya hidup digital (\textit{digital lifestyle}), Telkomsel turut membangun ekosistem digital di tanah air melalui berbagai upaya pengembangan DNA (\textit{Device}, \textit{Network} dan \textit{Applications}), yang diharapkan akan mempercepat terbentuknya masyarakat digital Indonesia.\\
\tab Sebelumnya, untuk pencatatan hasil pencapaian \textit{revenue} hanya dilakukan dengan menggunakan \textit{tools} berupa Microsoft Excel dan media \textit{WhatsApp} maupun \textit{E-mail} untuk menyebarkan hasilnya. Oleh karena itu, dibutuhkan suatu sistem untuk memantau pencapai \textit{revenue digital services} yang disediakan oleh Telkomsel\cite{telkomsel}.

\section{Tujuan}
Pembuatan Sistem TPORT ini bertujuan untuk:
\begin{enumerate}
\item Pengolahan data langsung melalui aplikasi web yang dibangun
\item Tersedianya informasi laporan hasil pencapaian tanpa harus disebarkan melalui \textit{WhatsApp} atau \textit{E-mail}
\item Melakukan efisiensi kerja 
\end{enumerate}

\section{Manfaat}
Manfaat dari pembuatan Sistem TPORT adalah sebagai berikut:

\begin{enumerate}
	\item Memudahkan proses pemantauan pencapaian \textit{revenue}
	\item Memudahkan penyebaran hasil karena sudah bisa dibuka pada aplikasi web yang dibangun
\end{enumerate}

\section{Rumusan Masalah}
Rumusan Masalah dari Kerja Praktik ini adalah sebagai berikut:

\begin{enumerate}
	\item Bagaimana menciptakan aplikasi web yang mudah dimengerti oleh pengguna?
	\item Bagaimana menciptakan aplikasi web TPORT sehingga mempermudah proses pemantauan pencapaian \textit{revenue}?
\end{enumerate}

\section{Lokasi dan Waktu Kerja Praktik}
\tab Lokasi kerja praktik berada di Grapari Pemuda dengan alamat Jalan Pemuda No 27-31, Embong Kaliasin, Genteng, Surabaya.\\
\tab Adapun kerja praktik dimulai pada tanggal 8 Januari 2018 hingga 9 Februari 2018 dengan hari kerja Senin sampai Jumat pukul 08.00 sampai dengan pukul 17.00 WIB (8 jam kerja dan 1 jam istirahat).

\section{Metodologi}
Metodologi dalam pembuatan buku Kerja Praktik ini meliputi:
\begin{enumerate}
	\item \textbf{Perumusan Masalah}\\
	Untuk mengetahui domain dan fungsionalitas, dijelaskan secara rinci bagaimana sistem yang harus dibuat. Penjelasan oleh pembimbing kerja praktik kali ini menghasilkan beberapa catatan mengenai gambaran cara kerja sistem dan rincian kebutuhan sistem. Setelah mendapatkan gambaran sistem, diskusi lebih lanjut dilakukan guna menentukan DBMS, bahasa pemrograman, dan framework yang dipakai dalam pembuatan sistem.
	
	\item \textbf{Studi Literatur}\\
	Pada tahap ini, setelah ditentukannya DBMS, bahasa pemrograman sampai dengan framework yang digunakan, dilakukan studi literatur lanjut mengenai bagaimana penggunaannya dalam membangun sistem sesuai yang diharapkan.
	
	Secara garis besar, untuk membuat TPORT digunakan bahasa pemrograman HTML, CSS, Javascript dan PHP untuk \textit{back end} sistem, serta DBMS MySQL sebagai penyimpanan data pencapaian \textit{revenue}, yang dikemas	melalui framework Laravel.
	
	\item \textbf{Analisis dan Perancangan Sistem}\\
	Pada tahap ini akan dijelaskan tentang analisis serta perancangan sistem yang akan dibangun oleh penulis.
	
	\item \textbf{Implementasi Sistem}\\
	Implementasi merupakan tahap pembangunan rancangan. Pada tahap ini merealisasikan apa yang terjadi pada tahap sebelumnya, sehingga sesuai dengan apa yang telah direncanakan.
	
	\item \textbf{Pengujian dan Evaluasi}\\
	Pada tahap ini dilakukan uji coba pada aplikasi yang telah diimplementasikan. Tahap ini bermaksud untuk mengevaluasi kesesuaian sistem dan aplikasi yang dibuat apakah dapat dilakukan dengan lancar atau tidak. Selain itu juga untuk mencari masalah yang mungkin timbul dan tidak lupa mengadakan perbaikan jika terdapat kesalahan.
	
	\item \textbf{Kesimpulan dan Saran}\\
	Pengujian yang dilakukan ini telah memenuhi syarat yang diinginkan, dan berjalan dengan baik dan lancar.
\end{enumerate}

\section{Sistematika Laporan}
Laporan Kerja Praktik ini terbagi menjadi 7 bab dengan rincian sebagai berikut:
	\begin{enumerate}
	\item BAB I: PENDAHULUAN
	
	Bab ini berisi latar belakang, tujuan, manfaat, rumusan masalah, lokasi dan waktu kerja praktik, metodologi dan sistematika laporan.
		
	\item BAB II: PROFIL PERUSAHAAN
		
	Bab ini berisi gambaran umum PT Telekomunikasi Seluler Indonesia, mulai dari sejarah, tujuan, visi dan misi perusahaan, dan divisi tempat kerja praktik dilakukan.
		
	\item BAB III: TINJAUAN PUSTAKA
		
	Bab ini berisi dasar teori dari metode/teknologi yang digunakan dalam menyelesaikan proyek kerja praktik.
		
	\item BAB IV: ANALISIS DAN PERANCANGAN SISTEM
		
	Bab ini dijelaskan mengenai desain antarmuka aplikasi.
		
	\item BAB V: IMPLEMENTASI SISTEM
		
	Bab ini berisi uraian tahap-tahap yang dilakukan untuk proses implementasi aplikasi.
		
	\item BAB VI: HASIL DAN UJI COBA
	
	Bab ini berisi hasil uji coba dan evaluasi dari perangkat lunak yang telah dikembangkan selama pelaksanaan kerja praktik.
	
	\item KESIMPULAN DAN SARAN
	
	Bab ini berisi kesimpulan dan saran yang didapat dari proses pelaksanaan kerja praktik.
	\end{enumerate}

\cleardoublepage

	\chapter{PROFIL PERUSAHAAN}
\tab Sejak berdiri pada tanggal 26 Mei 1995, Telkomsel secara konsisten melayani negeri, menghadirkan akses telekomunikasi kepada masyarakat Indonesia yang tersebar dari Sabang sampai Merauke. \\
\tab Saat ini Telkomsel adalah operator selular terbesar di Indonesia dengan 178 juta pelanggan dan untuk melayani pelanggannya yang tersebar di seluruh Indonesia, termasuk juga di daerah terpencil dan pulau terluar serta daerah perbatasan negara, Telkomsel menggelar lebih dari 146 ribu BTS. \\
\tab Telkomsel secara konsisten mengimplementasikan teknologi seluler terkini dan menjadi yang pertama meluncurkan secara komersial layanan mobile 4G LTE di Indonesia. Memasuki era digital, Telkomsel terus mengembangkan bisnis digital, diantaranya Digital Advertising, Digital Lifestyle, Mobile Financial Services, dan Internet of Things. Untuk melayani kebutuhan pelanggan, Telkomsel menggelar call center 24 jam dan layanan GraPARI yang tersebar di seluruh Indonesia.\cite{telkomsel}.
\section{Visi, Misi dan Tujuan Perusahaan}
Visi, Misi dan Tujuan dari Telkomsel adalah sebagai berikut:
	\subsection{Visi Perusahaan}
	Menjadi penyedia layanan dan solusi gaya hidup digital mobile kelas dunia yang terpercaya.
	\subsection{Misi Perusahaan}
	Memberikan layanan dan solusi digital mobile yang melebihi ekspektasi para pengguna, menciptakan nilai lebih bagi para pemegang saham serta mendukung pertumbuhan ekonomi bangsa.\cite{aboutus}

\section{Sejarah Perusahaan}
\tab Pada tahun 1993 PT Telkom mulai merambah teknologi nirkabel GSM, di tahun selanjutnya, pada 1994 PT Satelit Palapa Indonesia operator jaringan GSM pertama di Indonesia yang mengeluarkan kartu SIM muncul. PT Telkomsel kemudian didirikan bersama Indosat pada tahun 1995 dan meluncurkan kartu Halo pada tanggal 26 Mei 1995 sebagai layanan paska bayar. Pada tahun 2015 Saham Telkomsel dimiliki oleh Telkom Indonesia sebesar 65\% dan sisanya oleh Singtel sebesar 35\%.\\
\tab Telkomsel menjadi operator telekomunikasi seluler terbesar di Indonesia dengan 139,3 juta pelanggan per 31 Desember 2014 dan pangsa pasar sebesar 51\% per 1 Januari 2007.[butuh rujukan] Jaringan Telkomsel telah mencakup 288 jaringan roaming internasional di 155 negara pada akhir tahun 2007. \\
\tab Saat ini Telkomsel menggelar lebih dari 100.000 BTS yang menjangkau sekitar 98\% wilayah populasi di Indonesia. Sebagai operator seluler nomor 6 terbesar di dunia dalam hal jumlah pelanggan, Telkomsel merupakan pemimpin pasar industri telekomunikasi di Indonesia yang kini dipercaya melayani lebih dari 143 juta pelanggan pada tahun 2015-2016. Dalam upaya memandu perkembangan industri telekomunikasi seluler di Indonesia memasuki era baru layanan mobile broadband, Telkomsel secara konsisten mengimplementasikan roadmap teknologi 3G, HSDPA, HSPA+, serta pengembangan jaringan Long Term Evolution (LTE). Kini Telkomsel mengembangkan jaringan broadband di 100 kota besar di Indonesia. Untuk membantu pelayanan kebutuhan pelanggan, Telkomsel kini didukung akses call center 24 jam dan 430 pusat layanan yang tersebar di seluruh Indonesia. Telkomsel bekerja pada jaringan 900/1.800 MHz.\cite{sejarahtsel}.

\section{Divisi Digital Regional Expansion Jawa Timur}
\tab Pada kesempatan kali ini, penulis ditempatkan pada Divisi Digital Regional Expansion Jawa Timur. Divisi ini berhubungan dengan layanan-layanan digital yang disediakan oleh Telkomsel. Di sini, penulis berkesempatan membuat Sistem Informasi untuk Memantau Pencapaian \textit{Revenue}.

\cleardoublepage
	\chapter{TINJAUAN PUSTAKA}

\section{Laravel}
\tab Laravel adalah sebuah framework PHP yang dirilis dibawah lisensi MIT, dibangun dengan konsep MVC (model view controller). Laravel adalah pengembangan website berbasis MVP yang ditulis dalam PHP yang dirancang untuk meningkatkan kualitas perangkat lunak dengan mengurangi biaya pengembangan awal dan biaya pemeliharaan, dan untuk meningkatkan pengalaman bekerja dengan aplikasi dengan menyediakan sintaks yang ekspresif, jelas dan menghemat waktu\cite{laravel}.\\
\tab Dalam pengerjaan aplikasi Monitoring SIK ini, digunakan \textit{framework} laravel untuk memudahkan pembuatan website. Laravel merupakan sebuah \textit{framework} yang dapat mencukupi kebutuhan pengguna. Karena membutuhkan waktu yang singkat dalam pengerjaan (\textit{development}), serta mudah untuk \textit{maintenance} sistem.

\section{Javascript}
\tab Javascript adalah sebuah bahasa pemrograman tingkat tinggi yang dinamis yang terkenal dengan first class functionnya yang artinya javascript memperlakukan sebuah function dengan porsi yang sama dengan variabel lainnya. Contohnya adalah pada bahasa pemrograman javascript, sebuah function dapat dijadikan parameter dari function yang lainnya. Javascript merupakan bahasa scripting untuk halaman web yang paling populer saat ini. Selain lingkungan browser, banyak pula platform lain yang menggunakan bahasa javascript seperti node.js dan apache CouchDB\cite{javascript}.\\
\tab Dalam pengerjaan aplikasi Monitoring SIK, dibutuhkan beberapa halaman dengan fitur untuk menampilkan beberapa pilihan yang hanya bisa dipilih apabila memilih fitur tertentu, oleh karena itu digunakan bahasa javascript untuk menampilkan pilihan tersebut.

\section{PHP}
\tab PHP adalah bahasa pemrograman yang mengelola web service yang menggunakan protokol HTTP. Web Service ini dibuat agar bisa dipanggil atau diakses oleh aplikasi lain melalui internet dengan menggunakan format pertukaran data sebagai format pengiriman pesan. File PHP ini berisi query untuk mengolah database yang akan di proses pada aplikasi\cite{PHP}.

\section{Apache HTTP Server}
\tab Dalam pembuatan Sistem Monitoring SIK ini, digunakan Apache HTTP Server sebagai server yang menjalankan sistem yang telah dibuat. Apache HTTP Server merupakan sebuah \textit{paltform web server}. Apache mendukung beberapa fitur, beberapa diimplementasikan sebagai modul yang dikompilasi. Perangkat ini berkisar dari bahasa pemrograman sisi server yang mendukung skema otentikasi. Beberapa antarmuka bahasa yang mendukung antara lain Perl, Python, Tcl dan PHP\cite{apache}.

\section{MySQL }
\tab Database pada sistem ini menggunakan MySQL. MySQL adalah sebuah perangkat lunak sistem manajemen basis data SQL. Database MySQL mendukung beberapa fitur seperti \textit{multithread}, \textit{multi-user}, dan SQL database manajemen sistem(DMBMS). Database ini digunakan untuk keperluan sistem database yang cepat, handal dan mudah digunakan\cite{mysql}.
	\chapter{ANALISIS DAN PERANCANGAN SISTEM}
\tab Pada bab ini akan dijelaskan mengenai desain dan implementasi rancangan sistem informasi \textit{monitoring} SIK pada kegiatan relokasi ATM BRI. Sistem informasi \textit{Monitoring} SIK ini digunakan untuk \textit{memonitoring} jalannya kegiatan permintaan relokasi hingga selesai pengerjaan. Aplikasi ini dikhususkan untuk internal BRI dan provider penyedia jasa layanan dalam penanganan relokasi ATM BRI.

\section{Mengelola Request Relokasi}
Berikut adalah deskripsi sistem dan diagram aktivitas pada \textit{use case} Mengelola Request Relokasi. 
\subsection{Deskripsi Umum Sistem}
\tab Sistem akan melakukan pengelolaan request relokasi. Data yang dimasukkan akan disimpan ke dalam basis data. Pada kegiatan ini, terdapat sebuah parameter berupa kegiatan yang akan dilakukan pada saat relokasi ATM yaitu relokasi, instalasi dan reposisi.
\subsection{\textit{Use Case} dan Fitur Sistem}
Gambar \ref{figure:use_case_mengelola_req_relokasi} adalah \textit{use case} mengelola permintaan relokasi. Pada pengelolaan permintaan relokasi, user dapat melakukan beberapa kegiatan sebagai berikut:
	\subsubsection{Menambah Request Relokasi}
	User dapat menambahkan permintaan relokasi. Diagram \ref{figure:activity_menambah_req_relokasi} adalah diagram aktivitas menambah permintaan relokasi pada sistem \textit{Monitoring} SIK. Pada pengisian form permintaan relokasi ini, terdapat tiga buah kegiatan yaitu relokasi, instalasi dan reposisi. Pada kegiatan relokasi, user diharuskan memasukkan lokasi asal dan lokasi tujuan dari ATM yang akan di relokasi. Sedangkan pada kegiatan instalasi dan reposisi, hanya lokasi asal yang dapat di isi oleh user.
	\begin{figure}[h]
	\centerline {\includegraphics[width=9cm,height=6cm]{bab4/ActivityDiagram_MenambahkanReqRelokasi.png}}
	\caption{Diagram Aktivitas Menambah Request Relokasi}
	\label{figure:activity_menambah_req_relokasi}
	\end{figure}
		
	\subsubsection{Mengubah Request Relokasi}
	User dapat mengubah permintaan relokasi. Diagram \ref{figure:activity_mengubah_req_relokasi} adalah diagram aktivitas mengubah permintaan relokasi pada sistem \textit{Monitoring} SIK.
	\begin{figure}[h]
	\centerline {\includegraphics[width=9cm,height=5cm]{bab4/ActivityDiagram_MengubahReqRelokasi.png}}
	\caption{Diagram Aktivitas Mengubah Request Relokasi}
	\label{figure:activity_mengubah_req_relokasi}
	\end{figure}

	\subsubsection{Menghapus Request Relokasi}
	User dapat menghapus permintaan relokasi. Diagram \ref{figure:activity_menghapus_req_relokasi} adalah diagram aktivitas menghapus permintaan relokasi pada sistem \textit{Monitoring} SIK.
	\begin{figure}[h]
	\centerline {\includegraphics[width=9cm,height=5cm]{bab4/ActivityDiagram_MenghapusReqRelokasi.png}}
	\caption{Diagram Aktivitas Menghapus Request Relokasi}
	\label{figure:activity_menghapus_req_relokasi}
	\end{figure}		

	\begin{figure}[h]
	\centerline {\includegraphics[width=8cm,height=4.5cm]{bab4/use-case-mengelola-req-relokasi.png}}
	\caption{Diagram \textit{Use Case} Mengelola Request Relokasi}
	\label{figure:use_case_mengelola_req_relokasi}
	\end{figure}

\section{Mengelola SIK}
Berikut adalah deskripsi dan diagram aktivitas pada \textit{use case} Mengelola SIK.
\subsection{Deskripsi Umum Sistem}
\tab Sistem ini akan melakukan pengelolaan SIK(Surat Izin Kerja). Pembuatan SIK didasarkan pada permintaan relokasi yang ada pada daftar permintaan relokasi. Daftar-daftar permintaan relokasi yang telah dibuatkan SIK tidak akan dimunculkan kembali pada saat pengisian formulir penambahan SIK. Data yang dimasukkan akan disimpan ke dalam basis data.
\subsection{\textit{Use Case} dan Fitur Sistem}
Gambar \ref{figure:use_case_mengelola_sik} adalah \textit{use case} mengelola SIK. Pada pengelolaan SIK, user dapat melakukan beberapa kegiatan sebagai berikut:
	\subsubsection{Menambah SIK}
	User dapat menambahkan SIK berdasarkan permintaan relokasi yang ada. Batasan dalam pembuatan SIK yaitu setiap SIK harus dikerjakan oleh provider yang sama. Pada saat pembuatan SIK, permintaan relokasi yang telah dibuatkan SIK tidak akan ditampilkan ulang pada daftar permintaan relokasi yang ada. Diagram \ref{figure:activity_menambah_sik} adalah diagram aktivitas menambah SIK pada sistem \textit{Monitoring} SIK.
	\begin{figure}[h]
	\centerline {\includegraphics[width=9cm,height=6cm]{bab4/ActivityDiagram_MenambahkanSIK.png}}
	\caption{Diagram Aktivitas Menambah SIK}
	\label{figure:activity_menambah_sik}
	\end{figure}
		
	\subsubsection{Mengubah SIK}
	User dapat mengubah data SIK yang telah ditambahkan. Diagram \ref{figure:activity_mengubah_sik} adalah diagram aktivitas mengubah SIK pada sistem Monitoring SIK.
	\begin{figure}[h]
	\centerline {\includegraphics[width=9cm,height=5.6cm]{bab4/ActivityDiagram_MengubahSIK.png}}
	\caption{Diagram Aktivitas Mengubah SIK}
	\label{figure:activity_mengubah_sik}
	\end{figure}

	\subsubsection{Menghapus SIK}
	User dapat menghapus SIK. Pada penghapusan SIK, permintaan relokasi yang SIKnya dihapus akan ditampilkan ulang pada daftar permintaan relokasi. Diagram \ref{figure:activity_menghapus_sik} adalah diagram aktivitas menghapus SIK pada sistem \textit{Monitoring } SIK.
	\begin{figure}[h]
	\centerline {\includegraphics[width=9cm,height=4.5cm]{bab4/ActivityDiagram_MenghapusSIK.png}}
	\caption{Diagram Aktivitas Menghapus SIK}
	\label{figure:activity_menghapus_sik}
	\end{figure}		

	\begin{figure}[h]
	\centerline {\includegraphics[width=9cm,height=5cm]{bab4/use-case-mengelola-sik.png}}
	\caption{Diagram \textit{Use Case} Mengelola SIK}
	\label{figure:use_case_mengelola_sik}
	\end{figure}

\section{Mengelola Eksekusi}
Berikut adalah deskripsi dan diagram aktivitas pada \textit{use case} Mengelola Eksekusi.
\subsection{Deskripsi Umum Sistem}
\tab Sistem ini akan melakukan pengelolaan eksekusi permintaan relokasi. Pengelolaan eksekusi permintaan relokasi didasarkan pada SIK yang telah dibuat. Apabila SIK sudah masuk tahap \textit{approval} oleh kepala divisi, selanjutkan masuk pada tahap eksekusi.
\subsection{\textit{Use Case} dan Fitur Sistem}
Gambar \ref{figure:use_case_mengelola_eksekusi} adalah \textit{use case} mengelola eksekusi permintaan relokasi. Pada pengelolaan eksekusi permintaan relokasi, user dapat melakukan beberapa kegiatan sebagai berikut:
	\subsubsection{Menambah Eksekusi}
	User dapat menambahkan eksekusi permintaan relokasi berdasarkan pada SIK yang ada. Penambahan eksekusi dilakukan dengan pengubahan status SIK dari proses menjadi \textit{approved}. Diagram \ref{figure:activity_menambah_eksekusi} adalah diagram aktivitas menambah eksekusi request relokasi pada sistem \textit{Monitoring} SIK.
	\begin{figure}[h]
	\centerline {\includegraphics[width=9cm,height=6cm]{bab4/ActivityDiagram_MenambahkanEksekusi.png}}
	\caption{Diagram Aktivitas Menambah Eksekusi Request Relokasi}
	\label{figure:activity_menambah_eksekusi}
	\end{figure}
		
	\subsubsection{Mengubah Eksekusi}
	User dapat mengubah eksekusi permintaan relokasi yang telah dibuat. Perubahan pada data eksekusi akan disimpan menggantikan data yang sebelumnya telah dibuat. Diagram \ref{figure:activity_mengubah_eksekusi} adalah diagram aktivitas mengubah eksekusi permintaan relokasi pada sistem \textit{Monitoring} SIK.
	\begin{figure}[h]
	\centerline {\includegraphics[width=9cm,height=4.6cm]{bab4/ActivityDiagram_MengubahEksekusi.png}}
	\caption{Diagram Aktivitas Mengubah Eksekusi Request Relokasi}
	\label{figure:activity_mengubah_eksekusi}
	\end{figure}

	\subsubsection{Menghapus Eksekusi}
	User dapat menghapus eksekusi permintaan relokasi. Dan mengembalikan status eksekusi menjadi proses pembuatan SIK. Diagram \ref{figure:activity_menghapus_eksekusi} adalah diagram aktivitas menghapus eksekusi permintaan relokasi pada sistem \textit{Monitoring} SIK.
	\begin{figure}[h]
	\centerline {\includegraphics[width=9cm,height=4.5cm]{bab4/ActivityDiagram_MenghapusEksekusi.png}}
	\caption{Diagram Aktivitas Menghapus Eksekusi Request Relokasi}
	\label{figure:activity_menghapus_eksekusi}
	\end{figure}		

	\begin{figure}[h]
	\centerline {\includegraphics[width=9cm,height=5cm]{bab4/use-case-mengelola-eksekusi.png}}
	\caption{Diagram \textit{Use Case} Mengelola Eksekusi Request Relokasi}
	\label{figure:use_case_mengelola_eksekusi}
	\end{figure}

\section{Mengelola Penagihan}
Berikut adalah deskripsi dan diagram aktivitas pada \textit{use case} Mengelola Penagihan.
\subsection{Deskripsi Umum Sistem}
\tab Sistem ini akan melakukan pengelolaan penagihan permintaan relokasi. Pengelolaan penagihan permintaan relokasi didasarkan pada proses sebelumnya. Setelah proses eksekusi selesai, status SIK akan diubah menjadi penagihan pembayaran.
\subsection{\textit{Use Case} dan Fitur Sistem}
Gambar \ref{figure:use_case_mengelola_penagihan} adalah \textit{use case} mengelola penagihan permintaan relokasi. Pada pengelolaan penagihan permintaan relokasi, user dapat melakukan beberapa kegiatan sebagai berikut:
	\subsubsection{Menambah Penagihan}
	User dapat menambahkan penagihan permintaan relokasi. Berdasarkan pada proses sebelumnya, yaitu eksekusi. Penambahan proses penagihan dilakukan dengan pengubahan status SIK dari eksekusi menjadi penagihan. Diagram \ref{figure:activity_menambah_penagihan} adalah diagram aktivitas menambah penagihan permintaan relokasi pada sistem \textit{Monitoring} SIK.
	\begin{figure}[h]
	\centerline {\includegraphics[width=9cm,height=4.6cm]{bab4/ActivityDiagram_MenambahkanPenagihan.png}}
	\caption{Diagram Aktivitas Menambah Penagihan Request Relokasi}
	\label{figure:activity_menambah_penagihan}
	\end{figure}
	\subsubsection{Mengubah Penagihan}
	User dapat mengubah penagihan permintaan relokasi yang telah dibuat. Perubahan penagihan yang dibuat akan menggantikan data yang sebelumnya dibuat. Diagram \ref{figure:activity_mengubah_penagihan} adalah diagram aktivitas mengubah penagihan permintaan relokasi pada sistem Monitoring SIK.
	\begin{figure}[h]
	\centerline {\includegraphics[width=9cm,height=6cm]{bab4/ActivityDiagram_MengubahPenagihan.png}}
	\caption{Diagram Aktivitas Mengubah Penagihan Request Relokasi}
	\label{figure:activity_mengubah_penagihan}
	\end{figure}
	
	\subsubsection{Menghapus Penagihan}
	User dapat menghapus penagihan permintaan relokasi. proses penghapusan data penagihan akan mengembalikan proses penagihan menjadi proses eksekusi. Diagram \ref{figure:activity_menghapus_penagihan} adalah diagram aktivitas menghapus penagihan permintaan relokasi pada sistem \textit{Monitoring} SIK.
	\begin{figure}[h]
	\centerline {\includegraphics[width=9cm,height=4cm]{bab4/ActivityDiagram_MenghapusPenagihan.png}}
	\caption{Diagram Aktivitas Menghapus Penagihan Request Relokasi}
	\label{figure:activity_menghapus_penagihan}
	\end{figure}		

	\begin{figure}[h]
	\centerline {\includegraphics[width=9cm,height=5cm]{bab4/use-case-mengelola-penagihan.png}}
	\caption{Diagram \textit{Use Case} Mengelola Penagihan Request Relokasi}
	\label{figure:use_case_mengelola_penagihan}
	\end{figure}

\section{Melihat Request Relokasi}
Berikut adalah deskripsi dan diagram aktivitas pada \textit{use case} Melihat Request Relokasi.
\subsection{Deskripsi Umum Sistem}
\tab Sistem dapat menampilkan daftar permintaan relokasi kepada user berdasarkan \textit{record} yang tersimpan dalam basis data.
\subsection{\textit{Use Case} dan Fitur Sistem}
Gambar \ref{figure:use_case_melihat_req_relokasi} adalah \textit{use case} melihat daftar permintaan relokasi. Pada halaman melihat permintaan relokasi, user dapat melakukan beberapa kegiatan sebagai berikut:
	\subsubsection{Melihat Request Relokasi}
	User dapat melihat daftar permintaan relokasi yang telah ditambahkan ke dalam basis data. Diagram \ref{figure:activity_melihat_req_relokasi} adalah diagram aktivitas melihat daftar permintaan relokasi pada sistem \textit{Monitoring} SIK.
	\begin{figure}[h]
	\centerline {\includegraphics[width=9cm,height=4.5cm]{bab4/ActivityDiagram_MelihatReqRelokasi.png}}
	\caption{Diagram Aktivitas Melihat Daftar Request Relokasi}
	\label{figure:activity_melihat_req_relokasi}
	\end{figure}
	
	\subsubsection{Men\textit{download} Request Relokasi}
	User dapat men\textit{download} surat permintaan relokasi yang telah di\textit{upload} ke dalam basis data. Diagram \ref{figure:activity_mendownload_req_relokasi} adalah aktivitas diagram men\textit{download} surat permintaan relokasi pada sistem \textit{Monitoring} SIK.
	\begin{figure}[h]
	\centerline
	{\includegraphics[width=9cm,height=5cm]{bab4/ActivityDiagram_DownloadReqRelokasi.png}}
	\caption{Diagram Aktivitas Mendownload Request Relokasi}
	\label{figure:activity_mendownload_req_relokasi}
	\end{figure}

	\begin{figure}[h]
	\centerline
	{\includegraphics[width=9cm,height=5cm]{bab4/use-case-melihat-req-relokasi.png}}
	\caption{Diagram \textit{Use Case} Melihat Request Relokasi}
	\label{figure:use_case_melihat_req_relokasi}
	\end{figure}

\section{Melihat SIK}
Berikut adalah deskripsi dan diagram aktivitas pada \textit{use case} Melihat SIK.
\subsection{Deskripsi Umum Sistem}
\tab Sistem dapat menampilkan daftar SIK kepada user berdasarkan \textit{record} yang tersimpan dalam basis data.
\subsection{\textit{Use Case} dan Fitur Sistem}
Gambar \ref{figure:use_case_melihat_sik} adalah \textit{use case} melihat daftar SIK. Pada halaman melihat daftar SIK, user dapat melakukan beberapa kegiatan sebagai berikut:
	\subsubsection{Melihat SIK}
	User dapat melihat daftar SIK yang telah ditambahkan ke dalam basis data. Diagram \ref{figure:activity_melihat_sik} adalah diagram aktivitas melihat daftar SIK pada sistem \textit{Monitoring} SIK.
	\begin{figure}[h]
	\centerline {\includegraphics[width=9cm,height=5cm]{bab4/ActivityDiagram_MelihatSIK.png}}
	\caption{Diagram Aktivitas Melihat Daftar SIK}
	\label{figure:activity_melihat_sik}
	\end{figure}
	
	\subsubsection{Men\textit{download} SIK}
	User dapat men\textit{download} SIK yang telah di\textit{upload} ke dalam basis data. Diagram \ref{figure:activity_mendownload_sik} adalah aktivitas diagram men\textit{download} SIK pada sistem \textit{Monitoring} SIK.
	\begin{figure}[h]
	\centerline
	{\includegraphics[width=9cm,height=5cm]{bab4/ActivityDiagram_DownloadSIK.png}}
	\caption{Diagram Aktivitas Mendownload SIK}
	\label{figure:activity_mendownload_sik}
	\end{figure}

	\begin{figure}[h]
	\centerline
	{\includegraphics[width=9cm,height=5cm]{bab4/use-case-melihat-sik.png}}
	\caption{Diagram \textit{Use Case} Melihat SIK}
	\label{figure:use_case_melihat_sik}
	\end{figure}
	
\section{Melihat Eksekusi}
Berikut adalah deskripsi dan diagram aktivitas pada \textit{use case} Melihat Eksekusi.
\subsection{Deskripsi Umum Sistem}
\tab Sistem dapat menampilkan daftar eksekusi permintaan relokasi. Daftar eksekusi permintaan relokasi ditampilkan berdasarkan SIK yang memiliki status \textit{approval}.
\subsection{\textit{Use Case} dan Fitur Sistem}
Gambar \ref{figure:use_case_melihat_eksekusi} adalah \textit{use case} melihat daftar eksekusi permintaan relokasi. Pada halaman melihat daftar eksekusi permintaan relokasi, user dapat melakukan beberapa kegiatan sebagai berikut:
	\subsubsection{Melihat Ekekusi}
	User dapat melihat daftar eksekusi permintaan relokasi berdasarkan daftar SIK dengan status \textit{approval}. Diagram \ref{figure:activity_melihat_eksekusi} adalah diagram aktivitas melihat daftar eksekusi permintaan relokasi pada sistem \textit{Monitoring} SIK.
	\begin{figure}[h]
	\centerline {\includegraphics[width=9cm,height=4.6cm]{bab4/ActivityDiagram_MelihatEksekusi.png}}
	\caption{Diagram Aktivitas Melihat Daftar Eksekusi}
	\label{figure:activity_melihat_eksekusi}
	\end{figure}
	
	\subsubsection{Men\textit{download} Berita Acara}
	User dapat men\textit{download} berita acara yang telah di\textit{upload} ke dalam basis data. Diagram \ref{figure:activity_mendownload_berita_acara} adalah aktivitas diagram men\textit{download} berita acara pada sistem \textit{Monitoring} SIK.
	\begin{figure}[h]
	\centerline
	{\includegraphics[width=9cm,height=4.6cm]{bab4/ActivityDiagram_DownloadBeritaAcara.png}}
	\caption{Diagram Aktivitas Mendownload Berita Acara}
	\label{figure:activity_mendownload_berita_acara}
	\end{figure}
	
	\subsubsection{Men\textit{download} Tagihan}
	User dapat men\textit{download} penagihan pembayaran yang telah di\textit{upload} ke dalam basis data. Diagram \ref{figure:activity_mendownload_tagihan} adalah aktivitas diagram men\textit{download} penagihan pembayaran pada sistem \textit{Monitoring} SIK.
	\begin{figure}[h]
	\centerline
	{\includegraphics[width=9cm,height=5cm]{bab4/ActivityDiagram_DownloadTagihan.png}}
	\caption{Diagram Aktivitas Mendownload Tagihan Pembayaran}
	\label{figure:activity_mendownload_tagihan}
	\end{figure}

	\begin{figure}[h!]
	\centerline
	{\includegraphics[width=9cm,height=5cm]{bab4/use-case-melihat-eksekusi.png}}
	\caption{Diagram \textit{Use Case} Melihat Eksekusi}
	\label{figure:use_case_melihat_eksekusi}
	\end{figure}
	
\section{Melihat Penagihan}
Berikut adalah deskripsi dan diagram aktivitas pada \textit{use case} Melihat Penagihan.
\subsection{Deskripsi Umum Sistem}
\tab Sistem dapat menampilkan daftar penagihan pembayaran permintaan relokasi. Daftar penagihan pembayaran permintaan relokasi didasarkan pada proses eksekusi yang memiliki status \textit{finish}. Pada halaman ini, user dapat melihat proses-proses penagihan pembayaran dari kegiatan-kegiatan relokasi yang dikerjakan.
\subsection{\textit{Use Case} dan Fitur Sistem}
Gambar \ref{figure:use_case_melihat_penagihan} adalah \textit{use case} melihat daftar penagihan pembayaran permintaan relokasi. Pada halaman melihat daftar penagihan pembayaran permintaan relokasi, user dapat melakukan beberapa kegiatan sebagai berikut:
	\subsubsection{Melihat Penagihan}
	User dapat melihat daftar penagihan pembayaran permintaan relokasi berdasarkan daftar proses eksekusi dengan status \textit{finish}. Diagram \ref{figure:activity_melihat_penagihan} adalah diagram aktivitas melihat daftar penagihan pembayaran permintaan relokasi pada sistem \textit{Monitoring} SIK.
	
	\begin{figure}[h!]
	\centerline {\includegraphics[width=9cm,height=5cm]{bab4/ActivityDiagram_MelihatPenagihan.png}}
	\caption{Diagram Aktivitas Melihat Daftar Penagihan Pembayaran}
	\label{figure:activity_melihat_penagihan}
	\end{figure}

	\begin{figure}[h!]
	\centerline
	{\includegraphics[width=9cm,height=5.5cm]{bab4/use-case-melihat-penagihan.png}}
	\caption{Diagram \textit{Use Case} Melihat Penagihan Pembayaran}
	\label{figure:use_case_melihat_penagihan}
	\end{figure}
	
\section{Melihat Finish}
Berikut adalah deskripsi dan diagram aktivitas pada \textit{use case} Melihat Finish.
\subsection{Deskripsi Umum Sistem}
\tab Sistem dapat menampilkan daftar permintaan relokasi dengan status telah dibayar.
\subsection{\textit{Use Case} dan Fitur Sistem}
Gambar \ref{figure:use_case_melihat_finish} adalah \textit{use case} melihat daftar permintaan relokasi dengan status telah dibayar. Pada halaman melihat daftar permintaan relokasi dengan status telah dibayar, user dapat melakukan beberapa kegiatan sebagai berikut:
	\subsubsection{Melihat Finish}
	User dapat melihat daftar permintaan relokasi dengan status telah dibayar. Diagram \ref{figure:activity_melihat_finish} adalah diagram aktivitas melihat daftar penagihan pembayaran permintaan relokasi pada sistem \textit{Monitoring} SIK.
	\begin{figure}[h]
	\centerline {\includegraphics[width=9cm,height=5.5cm]{bab4/ActivityDiagram_MelihatFinish.png}}
	\caption{Diagram Aktivitas Melihat Daftar Permintaan Relokasi Sudah Dibayar}
	\label{figure:activity_melihat_finish}
	\end{figure}

	\begin{figure}[h!]
	\centerline
	{\includegraphics[width=9cm,height=5cm]{bab4/use-case-melihat-finish.png}}
	\caption{Diagram \textit{Use Case} Melihat Daftar Permintaan Relokasi Sudah Dibayar}
	\label{figure:use_case_melihat_finish}
	\end{figure}
	
\newpage
\section{Perancangan Data}
\tab Dalam pembuatan sebuah sistem informasi, database merupakan salah satu faktor utama yang penting. Perancangan dan alur data pada sebuah sistem harus dapat memenuhi kebutuhan-kebutuhan sistem, meliputi penambahan data, perubahan isi data dan penghapusan data dari basisdata sistem. Berikut adalah perancangan data pada sistem informasi \textit{monitoring} SIK.
\subsection{\textit{Conceptual Data Model}}
CDM (\textit{Conceptual Data Model}) merupakan sebuah model yang didasarkan pada objek-objek di dunia nyata. Objek dasar tersebut direpresentasikan dalam bentuk \textit{entity relationship diagram}. Gambar \ref{figure:CDM} adalah CDM pada pembuatan sistem informasi \textit{monitoring} SIK.
	\begin{figure}[h!]
	\centerline
	{\includegraphics[width=10cm,height=14cm]{bab4/CDM.png}}
	\caption{Diagram \textit{Conceptual Data Model}}
	\label{figure:CDM}
	\end{figure}
	
\subsection{\textit{Physical Data Model}}
PDM (\textit{Physical Data Model}) adalah model yang digambarkan dalam sebuah tabel serta hubungan-hubungan antara data dengan tabel lain. Gambar \ref{figure:PDM} adalah PDM pada pembuatan sistem informasi \textit{monitoring} SIK.
	\begin{figure}[h!]
	\centerline
	{\includegraphics[width=10cm,height=11cm]{bab4/PDM.png}}
	\caption{Diagram \textit{Physical Data Model}}
	\label{figure:PDM}
	\end{figure}
	\chapter{IMPLEMENTASI}
Pada bab ini akan dipaparkan implementasi dari sistem yang dibangun. Bahasa pemrograman yang digunakan adalah HTML, CSS, Javascript, dan PHP yang dikemas dalam framework Laravel.

\section{Lingkungan Implementasi}
\tab Lingkungan implementasi dalam pembuatan sistem pada kerja praktik kali ini meliputi perangkat keras dan perangkat lunak yang digunakan untuk mengimplementasikan sistem yang telah dirancang adalah sebagai berikut:
\begin{enumerate}
	\item Perangkat Keras
	\begin{itemize}
	\item \textit{Processor} Intel® Core™ i7-5500U CPU @ 2.40GHz
	\item Memori 4 GB
	\end{itemize}
	\item Perangkat Lunak
	\begin{itemize}
	\item Sistem Operasi Ubuntu 16.04 LTS 64 bit.
	\item \textit{Text editor} Visual Studio Code
	\item Bahasa pemrograman PHP.
	\end{itemize}
\end{enumerate}

\section{Tampilan Fitur Aplikasi TPORT}
Berikut adalah tampilan aplikasi TPORT.

\subsection{Tampilan Halaman \textit{Request} TPORT}
Tampilan halaman \textit{request} TPORT menggunakan bahasa pemrograman HTML, CSS, Javascript dan PHP untuk tampilan sistem. Penampilan data pada halaman ini bersifat dinamis. Semua fitur pada aplikasi TPORT didaftar secara rinci dan dikemas dalam tampilan statis. Kemudian ketika salah satu fitur dipilih oleh user, maka akan diarahkan pada halaman sesuai fitur yang dipilih. Gambar \ref{figure:requestTPORT} dan \ref{figure:detailRequestTPORT} adalah tampilan halaman dan \ref{lst:request} adalah potongan kode dari halaman \textit{request} TPORT.

\begin{figure}[h!]
	\centerline
	{\includegraphics[width=10cm,height=5cm]{bab5/tampilanRequest.png}}
	\caption{Halaman \textit{Request} TPORT}
	\label{figure:requestTPORT}
\end{figure}

\begin{figure}[h!]
	\centerline
	{\includegraphics[width=10.5cm,height=7.5cm]{bab5/detailTampilanRequest.png}}
	\caption{Detail \textit{Form} pada Halaman \textit{Request} TPORT}
	\label{figure:detailRequestTPORT}
\end{figure}

\lstinputlisting[language=PHP, firstline=1, lastline=21, firstnumber=1, caption=Potongan Kode Tampilan Halaman \textit{Request} TPORT, label={lst:request}]{bab5/src/halamanReport.php}

\subsection{Tampilan Halaman \textit{Upload} TPORT}
Pada halaman menambahkan \textit{upload} TPORT bahasa yang digunakan adalah HTML, CSS, PHP dan Javascript. Sebuah \textit{form} akan ditampilkan dan kemudian pengguna akan mengisikan data-data sesuai spesifikasi file yang diupload. File yang diupload berupa file .csv (\textit{Comma Separated Values}). \textit{Form} tersebut akan menampung data-data yang diperlukan, kemudian akan disimpan ke dalam basis data sistem. Gambar \ref{figure:uploadTPORT} dan \ref{figure:detailUploadTPORT} adalah tampilan halaman \textit{upload} dan \ref{lst:uploadTPORT} adalah potongan kode halaman \textit{upload} TPORT.

\begin{figure}[h!]
\centerline
{\includegraphics[width=10cm,height=5cm]{bab5/tampilanUpload.png}}
\caption{Halaman \textit{Upload} TPORT}
\label{figure:uploadTPORT}
\end{figure}

\begin{figure}[h!]
	\centerline
	{\includegraphics[width=10.5cm,height=7.5cm]{bab5/detailTampilanUpload.png}}
	\caption{Detail \textit{Form} pada Halaman \textit{Upload} TPORT}
	\label{figure:detailUploadTPORT}
\end{figure}


\lstinputlisting[language=PHP, firstline=1, lastline=59, firstnumber=1, caption=Potongan Kode Tampilan \textit{Upload} TPORT, label={lst:uploadTPORT}]{bab5/src/halamanUpload.php}

\subsection{Tampilan Halaman Target TPORT}
Pada halaman target TPORT bahasa yang digunakan adalah HTML, CSS, PHP dan Javascript. Sebuah \textit{form} akan ditampilkan kemudian pengguna mengisikan nilai \textit{revenue} sesuai target yang ingin diraih untuk tiap kluster. \textit{Form} tersebut akan menampung data-data yang diperlukan, kemudian akan disimpan ke dalam basis data sistem. Gambar \ref{figure:targetTPORT} dan \ref{figure:detailTargetTPORT} adalah tampilan halaman dan \ref{lst:targetTPORT} adalah potongan kode dari halaman target.

\lstinputlisting[language=PHP, firstline=1, lastline=43, firstnumber=1, caption=Potongan Kode Tampilan Target TPORT, label={lst:targetTPORT}]{bab5/src/halamanTarget.php}

\begin{figure}[h!]
	\centerline
	{\includegraphics[width=10cm,height=5cm]{bab5/tampilanTarget.png}}
	\caption{Halaman Target TPORT}
	\label{figure:targetTPORT}
\end{figure}

\begin{figure}[h!]
	\centerline
	{\includegraphics[width=10.5cm,height=6cm]{bab5/detailTampilanTarget.png}}
	\caption{Detail pada Halaman Target TPORT}
	\label{figure:detailTargetTPORT}
\end{figure}

\subsection{Halaman Pencapaian \textit{Revenue}}
Pada halaman pencapaian \textit{revenue} bahasa pemrograman yang digunakan adalah HTML, CSS dan PHP. Pada halaman ini, pengguna dapat melihat detail pencapaian \textit{revenue} berdasarkan wilayah, tanggal, dan kategori yang dipilih. Halaman ini akan menampilkan detail informasi sesuai pilihan pengguna tadi. Gambar \ref{figure:pencapaianTPORT} dan \ref{figure:detailPencapaianTPORT} adalah tampilan halaman dan \ref{lst:pencapaianTPORT} adalah potongan kode dari halaman detail pencapaian \textit{revenue}.

\begin{figure}[h!]
	\centerline
	{\includegraphics[width=10cm,height=5cm]{bab5/tampilanDetailPencapaian.png}}
	\caption{Halaman Detail Pencapaian \textit{Revenue}}
	\label{figure:pencapaianTPORT}
\end{figure}

\begin{figure}[h!]
	\centerline
	{\includegraphics[width=9cm,height=6cm]{bab5/detailTampilanPencapaian.png}}
	\caption{Detail Halaman Pencapaian \textit{Revenue}}
	\label{figure:detailPencapaianTPORT}
\end{figure}

\lstinputlisting[language=PHP, firstline=1, lastline=24, firstnumber=1, caption=Potongan Kode Tampilan Target TPORT, label={lst:pencapaianTPORT}]{bab5/src/halamanDetail.php}

	\chapter{HASIL DAN UJI COBA}
Pada bab ini akan dipaparkan hasil uji coba saat sistem dijalankan. Uji coba sistem TPORT akan dilakukan untuk memastikan kualitas perangkat lunak yang dikembangkan dengan analisis dan perancangan perangkat lunak.
\section{Lingkungan Uji Coba}
Lingkungan uji coba sistem pada Kerja Praktik kali ini meliputi perangkat keras dan perangkat lunak adalah sebagai berikut:
\begin{enumerate}
	\item Perangkat Keras
	\begin{itemize}
		\item \textit{Processor} Intel® Core™ i7-5500U CPU @ 2.40GHz
		\item Memori 4 GB
	\end{itemize}
	\item Perangkat Lunak
	\begin{itemize}
		\item Sistem Operasi Ubuntu 16.04 LTS 64 bit.
	\end{itemize}
\end{enumerate}

\section{Skenario Pengujian}
Skenario pengujian yang akan dilakukan pada aplikasi TPORT adalah melakukan peran sebagai admin yang sedang membuka aplikasi. Langkah-langkah dari skenario adalah sebagai berikut:
	\begin{enumerate}
	\item Pengguna membuka aplikasi TPORT
	\item Pengguna memilih menu \textit{upload}, \textit{request}, dan target
	\item Pengguna menambahkan data pencapaian \textit{revenue} dengan mengupload file pada halaman upload
	\item Pengguna melihat detail hasil pencapaian \textit{revenue} pada halaman \textit{request}
	\end{enumerate}
	
\subsection{Pengujian Mengupload Data Pencapaian \textit{Revenue}}
Pengujian ini dilakukan terhadap fungsionalitas \textit{upload} data pencapaian \textit{revenue}. Tabel \ref{tab:list_upload} menjelaskan pengujian fungsionalitas ini.

\begin{table}[h!]
	\centering
	\begin{tabular}{|p{4cm}|p{6cm}|}
	\hline
	Kode \textit{Use Case} & UC-001\\ \hline
	Tujuan Pengujian & \textit{Upload} semua data hasil pencapaian \textit{revenue}\\ \hline
	Data Masukan & File .csv \\ \hline
	Prosedur Pengujian & 
		\begin{enumerate}
		\item Pengguna \textit{login} sebagai administrator
		\item Pengguna memilih menu \textit{upload}
		\end{enumerate}\\ \hline
	Hasil yang diharapkan & Semua data hasil pencapaian \textit{revenue} dapat di-\textit{upload} dan data yang di-\textit{upload} dapat masuk ke basis data sistem \\ \hline
	Hasil yang diperoleh & Semua data hasil pencapaian \textit{revenue} dapat di-\textit{upload} dan masuk ke basis data sistem \\ \hline
	Kesimpulan & Proses \textit{upload} hasil pencapaian \textit{revenue} berhasil \\ \hline
	Kondisi Akhir & Data hasil pencapaian \textit{revenue} masuk ke basis data sistem\\ \hline
	\end{tabular}\caption{Skenario Pengujian \textit{Upload} Data Hasil Pencapaian \textit{Revenue}}
	\label{tab:list_upload}
\end{table}

\subsection{Pengujian Mengelola Target Pencapaian \textit{Revenue}}
Pengujian ini dilakukan terhadap fungsionalitas mengelola target pencapaian \textit{revenue}. Tabel \ref{tab:list_target} menjelaskan pengujian fungsionalitas ini. Gambar \ref{figure:lihatTarget} dan \ref{figure:detailLihatTarget} adalah hasil fungsionalitas menampilkan target serta mengubah target.

\begin{figure}[h!]
\centerline
{\includegraphics[width=10cm,height=5cm]{bab6/halamanTarget.png}}
\caption{Detail Data Target}
\label{figure:lihatTarget}
\end{figure}

\begin{figure}[h!]
	\centerline
	{\includegraphics[width=10cm,height=6cm]{bab6/detailHalamanTarget.png}}
	\caption{Detail Data Target}
	\label{figure:detailLihatTarget}
\end{figure}

\begin{table}[h!]
	\centering
	\begin{tabular}{|p{4cm}|p{6cm}|}
	\hline
	Kode \textit{Use Case} & UC-002\\ \hline
	Tujuan Pengujian & Menampilkan dan mengubah target pencapaian \textit{revenue}\\ \hline
	Data Masukan & - \\ \hline
	Prosedur Pengujian & 
		\begin{enumerate}
		\item Pengguna \textit{login} sebagai administrator
		\item Pengguna memilih menu target
		\end{enumerate}\\ \hline
	Hasil yang diharapkan & Semua data target pencapaian \textit{revenue} dapat ditampilkan serta dapat diubah pada menu target \\ \hline
	Hasil yang diperoleh & Semua data target pencapaian \textit{revenue} dapat ditampilkan dan diubah \\ \hline
	Kesimpulan & Proses menampilkan dan mengubah data target pencapaian \textit{revenue} berhasil\\ \hline
	Kondisi Akhir & Pengguna mendapatkan semua informasi data target pencapaian serta dapat mengubahnya \textit{revenue}\\ \hline
	\end{tabular}\caption{Skenario Pengujian Menampilkan dan Mengubah Data Target Pencapaian \textit{Revenue}}
	\label{tab:list_target}
\end{table}

\subsection{Pengujian Menampilkan Hasil Pencapaian \textit{Revenue}}
Pengujian ini dilakukan terhadap fungsionalitas menampilkan hasil pencapaian \textit{revenue}. Tabel \ref{tab:list_request} menjelaskan pengujian fungsionalitas ini. Gambar \ref{figure:requestL1} dan \ref{figure:detailRequestL1} adalah hasil fungsionalitas menampilkan pencapaian \textit{revenue} berdasarkan L1. Gambar \ref{figure:requestL3} dan \ref{figure:detailRequestL3} adalah hasil fungsionalitas menampilkan pencapaian \textit{revenue} berdasarkan L3. Gambar \ref{figure:requestTop5} \ref{figure:detailGrafikRequestTop5}, dan \ref{figure:detailTabelRequestTop5} adalah hasil fungsionalitas menampilkan pencapaian \textit{revenue} berdasarkan Top 5.

\begin{figure}[h!]
	\centerline
	{\includegraphics[width=10cm,height=5cm]{bab6/halamanL1.png}}
	\caption{Hasil Pencapaian \textit{Revenue} Berdasarkan L1}
	\label{figure:requestL1}
\end{figure}

\begin{figure}[h!]
	\centerline
	{\includegraphics[width=9cm,height=6cm]{bab6/detailHalamanL1.png}}
	\caption{Detail Hasil Pencapaian \textit{Revenue} Berdasarkan L1}
	\label{figure:detailRequestL1}
\end{figure}

\begin{figure}[h!]
	\centerline
	{\includegraphics[width=10cm,height=5cm]{bab6/halamanL3.png}}
	\caption{Hasil Pencapaian \textit{Revenue} Berdasarkan L3}
	\label{figure:requestL3}
\end{figure}

\begin{figure}[h!]
	\centerline
	{\includegraphics[width=9cm,height=7cm]{bab6/detailHalamanL3.png}}
	\caption{Detail Hasil Pencapaian \textit{Revenue} Berdasarkan L3}
	\label{figure:detailRequestL3}
\end{figure}

\begin{figure}[ph]
	\centerline
	{\includegraphics[width=10cm,height=5cm]{bab6/halamanT5.png}}
	\caption{Hasil Pencapaian \textit{Revenue} Berdasarkan Top 5}
	\label{figure:requestTop5}
\end{figure}

\begin{figure}[h!]
	\centerline
	{\includegraphics[width=10cm,height=5cm]{bab6/detailGrafikHalamanT5.png}}
	\caption{Detail Grafik Hasil Pencapaian \textit{Revenue} Berdasarkan Top 5}
	\label{figure:detailGrafikRequestTop5}
\end{figure}

\begin{figure}[h!]
	\centerline
	{\includegraphics[width=12cm,height=4cm]{bab6/detailTabelHalamanT5.png}}
	\caption{Detail Tabel Hasil Pencapaian \textit{Revenue} Berdasarkan Top 5}
	\label{figure:detailTabelRequestTop5}
\end{figure}

\begin{table}[h!]
	\centering
	\begin{tabular}{|p{4cm}|p{6cm}|}
		\hline
		Kode \textit{Use Case} & UC-003\\ \hline
		Tujuan Pengujian & Menampilkan hasil pencapaian \textit{revenue} berdasarkan L1, L3, dan Top 5\\ \hline
		Data Masukan & - \\ \hline
		Prosedur Pengujian & 
		\begin{enumerate}
			\item Pengguna \textit{login} sebagai administrator
			\item Pengguna memilih menu \textit{request}
		\end{enumerate}\\ \hline
		Hasil yang diharapkan & Semua hasil pencapaian \textit{revenue} berdasarkan L1, L3, dan Top5 dapat ditampilkan \\ \hline
		Hasil yang diperoleh & Semua hasil pencapaian \textit{revenue} berdasarkan L1, L3, dan Top5 dapat ditampilkan \\ \hline
		Kesimpulan & Proses menampilkan hasil pencapaian \textit{revenue} berdasarkan L1, L3, dan Top 5 berhasil\\ \hline
		Kondisi Akhir & Pengguna dapat melihat hasil pencapaian \textit{revenue} berdasarkan L1, L3, dan Top 5\\ \hline
	\end{tabular}\caption{Skenario Pengujian Menampilkan Hasil Pencapaian \textit{Revenue}}
	\label{tab:list_request}
\end{table}


\section{Evaluasi Pengujian}
\tab Hasil pengujian dihasilkan dari pengamatan lebih lanjut terhadap perilaku Sistem TPORT terhadap skenario kasus uji coba. Pengujian dilakukan \textit{internal team} untuk mencoba sistem yang telah diterapkan. Tabel \ref{tab:hasil_pengujian} adalah hasil pengujian pada Sistem TPORT.
\begin{table}[h!]
	\centering
	\begin{tabular}{|p{6cm}|p{4cm}|}
	\hline
	\textbf{Tugas} & \textbf{Hasil}\\ \hline
	Sistem mampu menyimpan informasi data pencapaian \textit{revenue} yang di-\textit{upload} oleh pengguna & Terpenuhi\\ \hline
	Sistem mampu menampilkan detail informasi target kepada pengguna & Terpenuhi\\ \hline
	Sistem mampu menangani pengelolaan data target & Terpenuhi\\ \hline
	Sistem mampu menampilkan hasil pencapaian \textit{revenue} & Terpenuhi\\ \hline
	Sistem menampilkan data sesuai dengan opsi yang dipilih oleh pengguna & Terpenuhi\\ \hline
	\end{tabular}\caption{Hasil Pengujian}
		\label{tab:hasil_pengujian}
\end{table}

Dengan hasil pengujian yang telah dilakukan, dapat disimpulkan bahwa keseluruhan Sistem TPORT memenuhi kriteria yang disebutkan pada sub-bab sebelumnya.

\section{Evaluasi Performa}
Tabel \ref{tab:evaluasi_performa} adalah hasil dari uji coba evaluasi performa Sistem TPORT.

\begin{table}
\centering
\begin{tabular}{|p{5cm}|p{2.5cm}|p{2cm}|}
\hline
\textbf{Tugas} & \textbf{Hasil} & \textbf{Waktu}\\ \hline
Membuka halaman login & Terpenuhi & 1 detik\\ \hline
Masuk sebagai pengguna & Terpenuhi & 1 detik\\ \hline
Menampilkan halaman menu & Terpenuhi & 1 detik\\ \hline
Membuka halaman upload & Terpenuhi & 1 detik\\ \hline
Melakukan proses \textit{upload} & Terpenuhi & 10 detik\\ \hline
Membuka halaman \textit{request} & Terpenuhi & 1 detik\\ \hline
Menampilkan hasil \textit{request} & Terpenuhi & 30 detik\\ \hline
Membuka halaman target & Terpenuhi & 1 detik\\ \hline
Menampilkan halaman target & Terpenuhi & 1 detik\\ \hline
\end{tabular}\caption{Hasil Uji Performa}
		\label{tab:evaluasi_performa}
\end{table}

\vspace{4 cm}
\textcolor{white}{..}
	\chapter{KESIMPULAN DAN SARAN}

\section{Kesimpulan}
Kesimpulan dari Kerja Praktik kali ini adalah sebagai berikut:
\begin{enumerate}
\item Aplikasi \textit{Monitoring} SIK berhasil dibuat dengan menggunakan bahasa pemrograman HTML, CSS, Javascript dan PHP untuk dapat memenuhi kebutuhan standar dalam pengelolaan permintaan relokasi ATM BRI.
\item Penggunaan \textit{framework} Laravel dan BDMS MySQL berhasil diterapkan untuk membuat aplikasi \textit{Monitoring} SIK, yang sangat diperlukan untuk memantau kegiatan pengelolaan permintaan relokasi hingga pembayaran kegiatan relokasi ATM BRI.
\item Performa dari sistem informasi \textit{Monitoring} SIK memiliki performa yang baik dalam penanganan tugas yang dikerjakan menggunakan sistem yang telah dibuat.
\end{enumerate}

\section{Saran}
Setelah melalui Kerja Praktik kali ini, penulis memberikan saran sebagai berikut:
\begin{enumerate}
\item Tampilan \textit{front end} harus teroptimasi dengan baik pada seluruh \textit{device} dan \textit{browser}. Mengingat beragamnya \textit{device} dan \textit{browser} yang digunakan pengguna.
\item Perlu dilakukan penelitian terhadap segala pengembangan fitur sistem di masa mendatang.
\end{enumerate}
	
	\appendix
	\begin{thebibliography}{9}
	
	\bibitem{PHP}
	\textbf{PHP, "What is PHP?"} [Online]. Available: http://php.net/manual/en/intro-whatis.php. [Accessed 4 October 2017].		
	
	\bibitem{javascript}
	\textbf{What is JavaScript?}, Mozilla Developer Network. [Online]. Tersedia pada: https://developer.mozilla.org/en-US/docs/Learn/JavaScript/First\_steps/
	What\_is\_JavaScript. [Accessed 6 October 2017].
	
	\bibitem{laravel}
	\textbf{What is Laravel?}, Laravel (Project Using Symphony) [Online]. Available: https://symfony.com/projects/laravel. [Accessed 8 October 2017].
	
	\bibitem{sejarahtsel}
	\textbf{Sejarah Telkomsel}, Sejarah Telkomsel [Online]. Available: https://id.wikipedia.org/wiki/Telkomsel. [Accessed 17 March 2018].
	
	\bibitem{telkomsel}
	\textbf {Sejarah Telkomsel}, Sejarah Kami [Online]. Available: https://www.telkomsel.com/about-us/our-story/our-history. [Accessed 17 March 2018].
	
	\bibitem{aboutus}
	\textbf{Tentang Telkomsel}, Tentang Telkomsel [Online]. Available: https://www.telkomsel.com/about-us/our-story. [Accesed 17 March 2018].
	
	\bibitem{apache}
	\textbf {Apache HTTP Server}, What is the Apache HTTP Server Project?, [Online]. Available: https://en.wikipedia.org/wiki/Apache\_HTTP\_Server. [Accessed 19 December 2017].
	
	\bibitem{mysql}
	\textbf{MySQL}, [Online]. Available: http://www.oracle.com/technetwork/database/mysql/index.html. [Accessed 19 December 2017].
\end{thebibliography}

	\backmatter
	\chapter{BIODATA PENULIS}
\begin{wrapfigure}{l}{0.4\textwidth}
\includegraphics[height=0.3\textheight]{biodata/hana.png}
\end{wrapfigure}

\textbf{Rohana Qudus}, lahir di Gresik tanggal 7 Februari 1998. Penulis merupakan anak kedua dari 2 bersaudara. Penulis telah menempuh pendidikan formal TK Dharmawanita Gresik, SD Muhammadiyah 2 Gresik (2004-2010), SMP Negeri 1 Gresik (2010-2013) dan SMA Negeri 1 Gresik (2013-2015). Penulis melanjutkan studi kuliah program sarjana di Departemen Informatika ITS. 

Selama berkuliah di Departemen Informatika ITS, penulis  pernah menjadi asisten dosen dan praktikum untuk mata kuliah Sistem Operasi (2017) dan Jaringan Komputer(2018). Selama menempuh perkuliahan penulis juga aktif di kegiatan organisasi dan kepanitiaan diantaranya menjadi Staf Departemen Media Informasi HMTC ITS, Staf Departemen Informasi Informasi Media BEM FTIF ITS, Staf Ahli Departemen Media Informasi HMTC ITS, Staf Website dan Kesekretariatan Schematics 2016, dan Staf Ahli 3D Schematics 2017. Penulis dapat dihubungi melalui surel di \\ \texttt{rohanaq27@gmail.com}.

\chapter{BIODATA PENULIS} 
\begin{wrapfigure}{l}{0.4\textwidth} 
	\includegraphics[height=0.3\textheight]{biodata/rafi.png} 
\end{wrapfigure} 

\textbf{Rafi R. Ramadhan}, lahir di Padang Panjang tanggal 27 Januari 1997. Penulis merupakan anak kedua dari 2 bersaudara. Penulis telah menempuh pendidikan formal TK Aisyiyah I Duri, Riau, SD IT Mutiara Duri (2003-2009), SMP IT Mutiara (2009-2012) dan SMA IT Mutiara Duri Riau (2012-2015). Penulis melanjutkan studi kuliah program sarjana di Jurusan Teknik Informatika ITS. 

Selama berkuliah di Departemen Informatika ITS, penulis aktif dalam kegiatan kemahasiswaan seperti Himpunan. Penulis merupakan staf pada Departemen Media dan Informasi HMTC 2016-2017, kemudian melanjutkan menjadi staf ahli pada departemen yang sama untuk periode 2017-2018. Selain di himpunan, penulis juga aktif pada Schematics selama 2 tahun berturut-turut. Penulis juga merupakan Volunteer dari International Office ITS. Penulis berada pada divisi Internationalization and Development Division, divisi ini mengemban tugas untuk mencerdaskan dan membantu proses internasionalisasi yang ada di kampus ITS. Penulis juga merupakan Leader dari Developer Student Clubs yang merupakan program gagasan dari Google Developer. penulis merupakan 1 dari 20 orang yang dipilih menjadi Leader untuk DSC se-Indonesia. Penulis dapat dihubungi melalui surel di \\ \texttt{rafi.ramadhan27@gmail.com}.
\end{document}
