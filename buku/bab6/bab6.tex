\chapter{HASIL DAN UJI COBA}
Pada bab ini akan dipaparkan hasil uji coba saat sistem dijalankan. Uji coba sistem TPORT akan dilakukan untuk memastikan kualitas perangkat lunak yang dikembangkan dengan analisis dan perancangan perangkat lunak.
\section{Lingkungan Uji Coba}
Lingkungan uji coba sistem pada Kerja Praktik kali ini meliputi perangkat keras dan perangkat lunak adalah sebagai berikut:
\begin{enumerate}
	\item Perangkat Keras
	\begin{itemize}
		\item \textit{Processor} Intel® Core™ i7-5500U CPU @ 2.40GHz
		\item Memori 4 GB
	\end{itemize}
	\item Perangkat Lunak
	\begin{itemize}
		\item Sistem Operasi Ubuntu 16.04 LTS 64 bit.
	\end{itemize}
\end{enumerate}

\section{Skenario Pengujian}
Skenario pengujian yang akan dilakukan pada aplikasi TPORT adalah melakukan peran sebagai admin yang sedang membuka aplikasi. Langkah-langkah dari skenario adalah sebagai berikut:
	\begin{enumerate}
	\item Pengguna membuka aplikasi TPORT
	\item Pengguna memilih menu \textit{upload}, \textit{request}, dan target
	\item Pengguna menambahkan data pencapaian \textit{revenue} dengan mengupload file pada halaman upload
	\item Pengguna melihat detail hasil pencapaian \textit{revenue} pada halaman \textit{request}
	\end{enumerate}
	
\subsection{Pengujian Mengupload Data Pencapaian \textit{Revenue}}
Pengujian ini dilakukan terhadap fungsionalitas \textit{upload} data pencapaian \textit{revenue}. Tabel \ref{tab:list_upload} menjelaskan pengujian fungsionalitas ini.

\begin{table}[h!]
	\centering
	\begin{tabular}{|p{4cm}|p{6cm}|}
	\hline
	Kode \textit{Use Case} & UC-001\\ \hline
	Tujuan Pengujian & \textit{Upload} semua data hasil pencapaian \textit{revenue}\\ \hline
	Data Masukan & File .csv \\ \hline
	Prosedur Pengujian & 
		\begin{enumerate}
		\item Pengguna \textit{login} sebagai administrator
		\item Pengguna memilih menu \textit{upload}
		\end{enumerate}\\ \hline
	Hasil yang diharapkan & Semua data hasil pencapaian \textit{revenue} dapat di-\textit{upload} dan data yang di-\textit{upload} dapat masuk ke basis data sistem \\ \hline
	Hasil yang diperoleh & Semua data hasil pencapaian \textit{revenue} dapat di-\textit{upload} dan masuk ke basis data sistem \\ \hline
	Kesimpulan & Proses \textit{upload} hasil pencapaian \textit{revenue} berhasil \\ \hline
	Kondisi Akhir & Data hasil pencapaian \textit{revenue} masuk ke basis data sistem\\ \hline
	\end{tabular}\caption{Skenario Pengujian \textit{Upload} Data Hasil Pencapaian \textit{Revenue}}
	\label{tab:list_upload}
\end{table}

\subsection{Pengujian Mengelola Target Pencapaian \textit{Revenue}}
Pengujian ini dilakukan terhadap fungsionalitas mengelola target pencapaian \textit{revenue}. Tabel \ref{tab:list_target} menjelaskan pengujian fungsionalitas ini. Gambar \ref{figure:lihatTarget} adalah hasil fungsionalitas menampilkan target serta mengubah target.

\begin{figure}[h!]
\centerline
{\includegraphics[width=10cm,height=5cm]{bab6/halamanTarget.png}}
\caption{Detail Data Target}
\label{figure:lihatTarget}
\end{figure}

\begin{table}[h!]
	\centering
	\begin{tabular}{|p{4cm}|p{6cm}|}
	\hline
	Kode \textit{Use Case} & UC-002\\ \hline
	Tujuan Pengujian & Menampilkan dan mengubah target pencapaian \textit{revenue}\\ \hline
	Data Masukan & - \\ \hline
	Prosedur Pengujian & 
		\begin{enumerate}
		\item Pengguna \textit{login} sebagai administrator
		\item Pengguna memilih menu target
		\end{enumerate}\\ \hline
	Hasil yang diharapkan & Semua data target pencapaian \textit{revenue} dapat ditampilkan serta dapat diubah pada menu target \\ \hline
	Hasil yang diperoleh & Semua data target pencapaian \textit{revenue} dapat ditampilkan dan diubah \\ \hline
	Kesimpulan & Proses menampilkan dan mengubah data target pencapaian \textit{revenue} berhasil\\ \hline
	Kondisi Akhir & Pengguna mendapatkan semua informasi data target pencapaian serta dapat mengubahnya \textit{revenue}\\ \hline
	\end{tabular}\caption{Skenario Pengujian Menampilkan dan Mengubah Data Target Pencapaian \textit{Revenue}}
	\label{tab:list_target}
\end{table}

\subsection{Pengujian Menampilkan Hasil Pencapaian \textit{Revenue}}
Pengujian ini dilakukan terhadap fungsionalitas menampilkan hasil pencapaian \textit{revenue}. Tabel \ref{tab:list_request} menjelaskan pengujian fungsionalitas ini. Gambar \ref{figure:requestL1}, \ref{figure:requestL3}, dan \ref{figure:requestTop5} adalah hasil fungsionalitas menampilkan hasil pencapaian \textit{revenue}.

\begin{figure}[h!]
	\centerline
	{\includegraphics[width=10cm,height=5cm]{bab6/halamanL1.png}}
	\caption{Hasil Pencapaian \textit{Revenue} Berdasarkan L1}
	\label{figure:requestL1}
\end{figure}

\begin{figure}[h!]
	\centerline
	{\includegraphics[width=10cm,height=5cm]{bab6/halamanL3.png}}
	\caption{Hasil Pencapaian \textit{Revenue} Berdasarkan L3}
	\label{figure:requestL3}
\end{figure}

\begin{figure}[h!]
	\centerline
	{\includegraphics[width=10cm,height=5cm]{bab6/halamanT5.png}}
	\caption{Hasil Pencapaian \textit{Revenue} Berdasarkan Top 5}
	\label{figure:requestTop5}
\end{figure}

\begin{table}[h!]
	\centering
	\begin{tabular}{|p{4cm}|p{6cm}|}
		\hline
		Kode \textit{Use Case} & UC-003\\ \hline
		Tujuan Pengujian & Menampilkan hasil pencapaian \textit{revenue} berdasarkan L1, L3, dan Top 5\\ \hline
		Data Masukan & - \\ \hline
		Prosedur Pengujian & 
		\begin{enumerate}
			\item Pengguna \textit{login} sebagai administrator
			\item Pengguna memilih menu \textit{request}
		\end{enumerate}\\ \hline
		Hasil yang diharapkan & Semua hasil pencapaian \textit{revenue} berdasarkan L1, L3, dan Top5 dapat ditampilkan \\ \hline
		Hasil yang diperoleh & Semua hasil pencapaian \textit{revenue} berdasarkan L1, L3, dan Top5 dapat ditampilkan \\ \hline
		Kesimpulan & Proses menampilkan hasil pencapaian \textit{revenue} berdasarkan L1, L3, dan Top 5 berhasil\\ \hline
		Kondisi Akhir & Pengguna dapat melihat hasil pencapaian \textit{revenue} berdasarkan L1, L3, dan Top 5\\ \hline
	\end{tabular}\caption{Skenario Pengujian Menampilkan Hasil Pencapaian \textit{Revenue}}
	\label{tab:list_request}
\end{table}


\section{Evaluasi Pengujian}
\tab Hasil pengujian dihasilkan pengamatan lebih lanjut terhadap perilaku sistem \textit{Monitoring} SIK terhadap skenario kasus uji coba. Pengujian dilakukan \textit{internal team} untuk mencoba sistem yang telah diterapkan. Tabel \ref{tab:hasil_pengujian} adalah hasil pengujian pada sistem informasi \textit{monitoring} SIK.
\begin{table}[h!]
	\centering
	\begin{tabular}{|p{6cm}|p{4cm}|}
	\hline
	\textbf{Tugas} & \textbf{Hasil}\\ \hline
	Sistem mampu menampilkan informasi fitur-fitur \textit{Monitoring} SIK & Terpenuhi\\ \hline
	Sistem mampu menampilkan detail informasi kepada user & Terpenuhi\\ \hline
	Sistem mampu menangani pengelolaan data \textit{Monitoring} SIK & Terpenuhi\\ \hline
	Sistem mampu mengeksport data SIK ke dalam bentuk file berekstensi pdf & Terpenuhi\\ \hline
	Sistem menampilkan data sesuai dengan \textit{keyword} pencarian & Terpenuhi\\ \hline
	Sistem dapat mendownload dan upload file ke dalam basis data & Terpenuhi\\ \hline
	\end{tabular}\caption{Hasil Pengujian}
		\label{tab:hasil_pengujian}
\end{table}

Dengan hasil pengujian yang telah dilakukan, dapat disimpulkan bahwa keseluruhan aplikasi \textit{Monitoring} SIK memenuhi kriteria yang disebutkan pada sub-bab sebelumnya.

\section{Evaluasi Performa}
Tabel \ref{tab:evaluasi_performa_1} dan \ref{tab:evaluasi_performa_2} adalah hasil dari uji coba evaluasi performa sistem informasi \textit{monitoring} SIK.
\begin{table}
\centering
\begin{tabular}{|p{5cm}|p{2.5cm}|p{2cm}|}
\hline
\textbf{Tugas} & \textbf{Hasil} & \textbf{Waktu}\\ \hline
Membuka Halaman Dashboard & Terpenuhi & 2 detik\\ \hline
Membuka Halaman Request Relokasi & Terpenuhi & 1 detik\\ \hline
Menambahkan Request Relokasi & Terpenuhi & 2 detik\\ \hline
\end{tabular}\caption{Hasil Uji Performa(1)}
		\label{tab:evaluasi_performa_1}
\end{table}

\begin{table}
\centering
\begin{tabular}{|p{5cm}|p{2.5cm}|p{2cm}|}
\hline
\textbf{Tugas} & \textbf{Hasil} & \textbf{Waktu}\\ \hline
Membuka Halaman SIK & Terpenuhi & 1 detik\\ \hline
Menambahkan SIK & Terpenuhi & 2 detik\\ \hline
Membuka Halaman Penagihan & Terpenuhi & 1 detik\\ \hline
Menambahkan Penagihan & Terpenuhi & 2 detik\\ \hline
Membuka Halaman Eksekusi & Terpenuhi & 1 detik\\ \hline
Menambahkan Eksekusi & Terpenuhi & 2 detik\\ \hline
Membuka Halaman Finish & Terpenuhi & 1 detik\\ \hline
Menambahkan Finish & Terpenuhi & 2 detik\\ \hline
Mendownload Request Relokasi & Terpenuhi & 2 detik\\ \hline
Mendownload SIK & Terpenuhi & 2 detik\\ \hline
Mendownload Berita Acara & Terpenuhi & 2 detik\\ \hline
Mendownload Penagihan & Terpenuhi & 2 detik\\ \hline
\end{tabular}\caption{Hasil Uji Performa(2)}
		\label{tab:evaluasi_performa_2}
\end{table}

\vspace{4 cm}
\textcolor{white}{..}