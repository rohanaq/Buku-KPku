\chapter{PROFIL PERUSAHAAN}
\tab Sejak berdiri pada tanggal 26 Mei 1995, Telkomsel secara konsisten melayani negeri, menghadirkan akses telekomunikasi kepada masyarakat Indonesia yang tersebar dari Sabang sampai Merauke. \\
\tab Saat ini Telkomsel adalah operator selular terbesar di Indonesia dengan 178 juta pelanggan dan untuk melayani pelanggannya yang tersebar di seluruh Indonesia, termasuk juga di daerah terpencil dan pulau terluar serta daerah perbatasan negara, Telkomsel menggelar lebih dari 146 ribu BTS. \\
\tab Telkomsel secara konsisten mengimplementasikan teknologi seluler terkini dan menjadi yang pertama meluncurkan secara komersial layanan mobile 4G LTE di Indonesia. Memasuki era digital, Telkomsel terus mengembangkan bisnis digital, diantaranya Digital Advertising, Digital Lifestyle, Mobile Financial Services, dan Internet of Things. Untuk melayani kebutuhan pelanggan, Telkomsel menggelar call center 24 jam dan layanan GraPARI yang tersebar di seluruh Indonesia.\cite{telkomsel}.
\section{Visi, Misi dan Tujuan Perusahaan}
Visi, Misi dan Tujuan dari Telkomsel adalah sebagai berikut:
	\subsection{Visi Perusahaan}
	Menjadi penyedia layanan dan solusi gaya hidup digital mobile kelas dunia yang terpercaya.
	\subsection{Misi Perusahaan}
	Memberikan layanan dan solusi digital mobile yang melebihi ekspektasi para pengguna, menciptakan nilai lebih bagi para pemegang saham serta mendukung pertumbuhan ekonomi bangsa.\cite{aboutus}

\section{Sejarah Perusahaan}
\tab Pada tahun 1993 PT Telkom mulai merambah teknologi nirkabel GSM, di tahun selanjutnya, pada 1994 PT Satelit Palapa Indonesia operator jaringan GSM pertama di Indonesia yang mengeluarkan kartu SIM muncul. PT Telkomsel kemudian didirikan bersama Indosat pada tahun 1995 dan meluncurkan kartu Halo pada tanggal 26 Mei 1995 sebagai layanan paska bayar. Pada tahun 2015 Saham Telkomsel dimiliki oleh Telkom Indonesia sebesar 65\% dan sisanya oleh Singtel sebesar 35\%.\\
\tab Telkomsel menjadi operator telekomunikasi seluler terbesar di Indonesia dengan 139,3 juta pelanggan per 31 Desember 2014 dan pangsa pasar sebesar 51\% per 1 Januari 2007.[butuh rujukan] Jaringan Telkomsel telah mencakup 288 jaringan roaming internasional di 155 negara pada akhir tahun 2007. \\
\tab Saat ini Telkomsel menggelar lebih dari 100.000 BTS yang menjangkau sekitar 98\% wilayah populasi di Indonesia. Sebagai operator seluler nomor 6 terbesar di dunia dalam hal jumlah pelanggan, Telkomsel merupakan pemimpin pasar industri telekomunikasi di Indonesia yang kini dipercaya melayani lebih dari 143 juta pelanggan pada tahun 2015-2016. Dalam upaya memandu perkembangan industri telekomunikasi seluler di Indonesia memasuki era baru layanan mobile broadband, Telkomsel secara konsisten mengimplementasikan roadmap teknologi 3G, HSDPA, HSPA+, serta pengembangan jaringan Long Term Evolution (LTE). Kini Telkomsel mengembangkan jaringan broadband di 100 kota besar di Indonesia. Untuk membantu pelayanan kebutuhan pelanggan, Telkomsel kini didukung akses call center 24 jam dan 430 pusat layanan yang tersebar di seluruh Indonesia. Telkomsel bekerja pada jaringan 900/1.800 MHz.\cite{sejarahtsel}.

\section{Divisi Digital Regional Expansion Jawa Timur}
\tab Pada kesempatan kali ini, penulis ditempatkan pada Divisi Digital Regional Expansion Jawa Timur. Divisi ini berhubungan dengan layanan-layanan digital yang disediakan oleh Telkomsel. Di sini, penulis berkesempatan membuat Sistem Informasi untuk Memantau Pencapaian \textit{Revenue}.

\cleardoublepage